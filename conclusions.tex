\section{Conclusions}
\label{section:conclusion}

We performed a dynamo simulation in a rotating spherical shell to investigate processes of the reversal of the dipole component of the magnetic field. In the present study, we investigate the equatorial symmetry of the energy fluxes for the convection during the dipole reversal from the simulation results. We perform the simulation to approximately 90 times of the magnetic diffusion time and obtained 12 reversals and 4 excursions.

First, we investigate the characteristics of the temperature field during one reversal. The results show that the equatorially anti-symmetric temperature component becomes larger than the symmetric equatorially component, and that the intense upward flow is generated in the southern hemisphere in the tangent cylinder. Looking at the power spectrum of the kinetic energy and temperature as a function of spherical harmonics order $m$ during the reversal, the equatorially anti-symmetric and axisymmetric component of the toroidal kinetic energy and temperature increase significantly during the reversal. These results suggests the todoidal anti-symmetric zonal flow is generated with sustaining the therma wind balance around the hot upwelling flow around the tangent cylinder.

We investigate the equatorial symmetry of the energy fluxes for the convection during the dipole reversal from the simulation results. The energy fluxes changes the following during the reversal: i) The energy transfer from the equatorially symmetric kinetic energy by the Lorentz force decreases, ii) energy transfer from symmetric to anti-symmetric kinetic energies are increase, and then buoyancy flux to the anti-symmetric component of the kinetic energy increases. Looking at the amplitude of the change of the energy flux to/from anti-symmetric kinetic energy, the work of the inertia term was the largest, and the second largest was the buoyancy flux. These results are the same for the 10 of 11 periods including the reversal and excursion. Considering the change of the kinetic energy during the dipole reversal, The change of the work of the inertia increase the equatorially anti-symmetric and axissymmetric component of the flow around the tangent cylinder. These results suggested that intense zonal flow is only generated in the either hemispher, and generated reversed magnetic field from th
e exsited dipolar magnetic field.