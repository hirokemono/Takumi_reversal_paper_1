\section{Conclusions}
\label{section:conclusions}

We performed numerical simulations of dynamo in a rotating spherical shell for approximately 85 magnetic diffusion times and obtained 12 reversals and 4 excursions to understand the process of the polarity reversal of the magnetic field. 
We investigated the energy transfer for the convection during polarity reversals with special emphasis on symmetry with respect to the equatorial plane.

First, we examined the characteristics of the temperature field during one polarity reversal. 
We found that the equatorially antisymmetric temperature becomes larger than the equatorially symmetric one, and that the intense upward flow is generated inside the tangent cylinder in the southern hemisphere. 
Next, we examined the spectra of the kinetic energy and square of temperature as a function of the spherical harmonic order, $m$, %during polarity reversals. 
{\color{teal}
during stable and reversal periods.}
The equatorially antisymmetric toroidal flow and temperature with the axial symmetry are found to increase significantly during polarity reversals.
These results suggest that the equatorially antisymmetric zonal toroidal flow is generated to sustain the thermal wind balance around the hot upwelling flow in the tangent cylinder.

The energy transfer during the polarity reversal changes in the following way: 
(i) the energy transfer from the kinetic energy for the equatorially symmetric flow by the Lorentz force decreases, 
(ii) the energy transfer from the equatorially symmetric to antisymmetric kinetic energies increases, and (iii) the energy transfer to the kinetic energy for the equatorially antisymmetric flow by the buoyancy increases. 
The change of the energy transfer to/from the kinetic energy for the equatorially antisymmetric flow indicates that the work by inertia is the largest, and the buoyancy follows it. 
These results are common among the 10 of 11 periods during which polarity reversals and excursions occurred. 
The change of the kinetic energy during the polarity reversal shows that the work by the inertia contributes to increase of kinetic energy for the equatorially antisymmetric flows with the axial symmetry around the tangent cylinder. 
These results suggest that the intense zonal flow caused by the intense upward flow inside the tangent cylinder in the either hemisphere can trigger a polarity reversal of the magnetic field.

\section*{Acknowledgments}
The present study was supported by NSF EAR-1550901 (Computational Infrastructure for Geodynamics).
This study used the computational resources of the HPCI system provided by the Research Institute for Information Technology, Kyushu University through the HPCI System Research Project (Project IDs: hp210022, hp220040, hp230046 and hp240018), the Polar Science Computer System in the Communications and Computing science Center (CCC) of the National Institute of Polar Research (NIPR), and Frontera High Performance Computing resource in the Texas Advanced Computing Center (TACC). 
