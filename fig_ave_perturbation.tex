\begin{figure}[ht]
\begin{center}
\[
\begin{array}{c}
\includegraphics*[width=120mm]{Figures/Averaged_flux_perturbations_2.pdf}
\end{array}
\]
\end{center}
\caption{
 Time and volume average of difference of energy transfers in reversal periods from those in the stable period for calculations in Period 4 to 8 and re-calculations in the Period 1 to 3.
% {\color{blue}
$(\bvec{u}^s \cdot \bvec{F}_{L})$, $(\bvec{u}^s \cdot \bvec{F}_{I})$, and $(\bvec{u}^s \cdot \bvec{F}_{B})$ indicate energy transfers into equatorially symmetric components of the kinetic energy by Lorentz force $(Pm E)^{-1} \bvec{u}^s \cdot (\bvec{J} \times \bvec{B})$, advection $-\bvec{u}^s \cdot(\bvec{\omega} \times \bvec{u})$, and buoyancy flux $Ra E^{-1} \bvec{u}^s \cdot \bvec{r} T$, respectively. 
$(\bvec{u}^a \cdot \bvec{F}_{L})$, $(\bvec{u}^a \cdot \bvec{F}_{I})$, and $(\bvec{u}^a \cdot \bvec{F}_{B})$ indicate energy transfers into equatorially antisymmetric components of the kinetic energy by Lorentz force $(Pm E)^{-1} \bvec{u}^a \cdot (\bvec{J} \times \bvec{B})$, advection $-\bvec{u}^a \cdot(\bvec{\omega} \times \bvec{u})$, and buoyancy flux $Ra E^{-1} \bvec{u}^a \cdot \bvec{r} T$, respectively. 
% }
  %$(\bvec{u}^s \cdot \bvec{F}_{L})$ and $(\bvec{u}^a \cdot \bvec{F}_{L})$ indicate energy fluxes into equatorilly symmetric and antisymmetric components by Lorentz force $(Pm E)^{-1} \bvec{u}^s \cdot (\bvec{J} \times \bvec{B})$ and $(Pm E)^{-1} \bvec{u}^a \cdot (\bvec{J} \times \bvec{B})$, respectively. 
 %$(\bvec{u}^a \cdot \bvec{F}_{I})$ is the energy flux to equatorially antisymmetric components by advection term $-\bvec{u}^a \cdot(\bvec{\omega} \times \bvec{u})$. $(\bvec{u}^s \cdot \bvec{F}_{B})$ and $(\bvec{u}^a \cdot \bvec{F}_{B})$ indicate buoyancy flux for the equatorially symmetric and antisymmetric components $Ra E^{-1} \bvec{u}^s \cdot \bvec{r} T$ and $Ra E^{-1} \bvec{u}^a \cdot \bvec{r} T$, respectively.
}
\label{Fig:Change_flux_summary_6grp}
\end{figure}