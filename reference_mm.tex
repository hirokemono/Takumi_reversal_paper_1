\noindent
{\bf References}
%
\begin{list}
{}{
\setlength{\parsep}{0pt}
\setlength{\itemsep}{0pt}
\setlength{\leftmargin}{1.0em}
\setlength{\itemindent}{-\leftmargin}
}
% \bibitem[\protect\citename{Busse, }1970] {Busse:1970}
\item
Busse, F.H., 1970. Thermal instabilities in rapidly rotating systems. J. Fluid Mech. 44(3), 441--460.
% \bibitem[\protect\citename{Coe and Glatzmaier, }2006] {Coe:2006}
\item
Coe, R.S., Glatzmaier, G.A., 2006. Symmetry and stability of the geomagnetic field. Geophys.\ Res.\ Lett. 33(21), doi:10.1029/2006GL027903.
%
%\bibitem[\protect\citename{Glatzmaier et al., }1999] {Glatzmaier:1999}
\item
Glatzmaier, G.A., Coe, R.S., Hongre, L., Roberts, P.H., 1999. The role of the Earth's mantle in controlling the frequency of geomagnetic reversals. Nature 401(6756), 885--890.
%
%\bibitem[\protect\citename{Glatzmaier and Roberts, }1995] {GR:1995}
\item
Glatzmaier, G.A., Roberts, P.H., 1995. A three-dimensional self-consistent computer simulation of a geomagnetic field reversal. Nature 377(6546), 203--209.
%
%\bibitem[\protect\citename{Kageyama and Sato, }1997] {Kageyama:1997}
\item
Kageyama, A., Sato, T., 1997. Generation mechanism of a dipole field by a magnetohydrodynamic dynamo. Phys.\ Rev.\ E 55(4), 4617--4626.
%
%\bibitem[\protect\citename{Kageyama et al., }1995] {Kageyama:1995}
\item
Kageyama, A., Sato, T., the Complexity Simulation Group, 1995. Computer simulation of a magnetohydrodynamic dynamo, II. Phys.\ Plasmas 2(5), 1421--1431.
%
%\bibitem[\protect\citename{Li et al., }2002] {Li:2002}
\item
Li, J., Sato, T., Kageyama, A., 2002. Repeated and sudden reversals of the dipole field generated by a spherical dynamo action. Science 295(5561), 1887--1890.
%
%\bibitem[\protect\citename{Liu and Olson, }2009] {Liu:2009}
\item
Liu, L., Olson, P., 2009. Geomagnetic dipole moment collapse by convective mixing in the core. Geophys.\ Res.\ Lett. 36(10), doi:10.1029/2009GL038130.
%
%\bibitem[\protect\citename{Matsui et al., }2014] {Matsui:2014}
\item
Matsui, H., King, E., Buffett, B., 2014. Multiscale convection in a geodynamo simulation with uniform heat flux along the outer boundary. Geochem.\ Geophys.\ Geosys. 15(8), 3212--3225, doi:10.1002/2014GC005432.
%
%\bibitem[\protect\citename{McFadden et al., }1991] {McFadden:1991}
\item
McFadden, P.L., Merrill, R.T., McElhinny, M.W., Lee, S., 1991. Reversals of the Earth's magnetic field and temporal variations of the dynamo families. J. Geophys. Res. 96(B3), 3923--3933.
%
%\bibitem[\protect\citename{Nishikawa and Kusano, }2008] {Nishikawa:2008}
\item
Nishikawa, N., Kusano, K., 2008. Simulation study of the symmetry-breaking instability and the dipole field reversal in a rotating spherical shell dynamo. Phys.\ Plasmas 15(8), 082903.
%
%\bibitem[\protect\citename{Ogg et al., } 2005] {Ogg:2005}
\item
Ogg, J.G., Agterberg, F.P., Gradstein, F.M., 2005. The Cretaceous period. In Gradstein, F.M., Ogg, J.G., Smith, A.G., editors, A Geologic Time Scale 2004, pp.\ 344--383, Cambridge University Press, Cambridge.
%
%\bibitem[\protect\citename{Olson et al., }2011] {Olson:2011}
\item
\sloppy
Olson, L.P., Glatzmaier, G.A., Coe, R.S., 2011. Complex polarity reversals in a geodynamo model. Earth Planet.\ Sci.\ Lett.\ 304(1)--(2), 168--179, doi:10.1016/j.epsl.2011.01.031.
%
%\bibitem[\protect\citename{Sarson and Jones, }1999] {Sarson:1999}
\item
Sarson, G., Jones, C., 1999. A convection driven geodynamo reversal model. Phys. Earth Planet.\ Inter. 111, 3--20.
%
%\bibitem[\protect\citename{Sreenivasan et al., }2014] {Sreenivasan:2014}
\item
Sreenivasan, B., Sahoo, S., Dhama, G., 2014. The role of buoyancy in polarity reversals of the geodynamo. Geophys.\ J. Int.\ 199(3), 1698--1709.
%
%\bibitem[\protect\citename{Takahashi et al., }2007] {TMH:2007}
\item
Takahashi, F., Matsushima, M., Honkura, Y., 2007. A numerical study on mangnetic polarity transition in an MHD dynamo model. Earth Planets Space 59(7), 665--673.
%
%\bibitem[\protect\citename{Tarduno et al., }2020] {Tarduno:2020}
\item
Tarduno, J.A., Cottrell, R.D., Bono, R.K., Oda, H., Davis, W.J., Fayek, M., van't Erve, O., Nimmo, F., Huang, W., Thern, E.R., Fearn, S., Mitra, G., Smirnov, A.V., Blackman, E.G., 2020. Paleomagnetism indicates that primary magnetite in zircon records a strong Hadean geodynamo. Proc.\ Nat.\ Acad.\ Sci. 117(5), 2309.
%
%\bibitem[\protect\citename{Wicht and Olson, }2004] {Wicht:2004}
\item
Wicht, J. and Olson, P., 2004. A detailed study of the polarity reversal mechanism in a numerical dynamo model: reversal mechanism. Geochem.\ Geophys.\ Geosys. 5(3), doi:10.1029/2003GC000602.
%
%\bibitem[\protect\citename{Winch et al., }2005] {Winch:2005}
\item
Winch, D.E., Ivers, D.J., Turner, J.P.R. and Stening, R. J., 2005. Geomagnetism and Schmidt quasi-normalization. Geophys. J. Int. 160(2), 487--504.
%
\end{list}