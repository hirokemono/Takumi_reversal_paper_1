\bigskip
\noindent
{\bf References}
%
\begin{list}
{}{
\setlength{\parsep}{0pt}
\setlength{\itemsep}{0pt}
\setlength{\leftmargin}{1.0em}
\setlength{\itemindent}{-\leftmargin}
}
% {\color{red}
\item 
\sloppy
Amit, H., Leonhardt, R., Wicht, J., 2010. Polarity reversals from paleomagnetic observations and numerical dynamo simulations. Space Sci.\ Rev.\ 155(1--4), 293--335. https://doi.org/10.1007/s11214-010-9695-2.
% }
%
\item
\sloppy
Aubert, J., 2019. Approaching Earth's core conditions in high-resolution geodynamo simulations. Geophys.\ J. Int.\ 219 (Supplement 1), S137--S151. https://doi.org/10.1093/gji/ggz232.
%
% \bibitem[\protect\citename{Busse, }1970] {Busse:1970}
\item
Busse, F.H., 1970. Thermal instabilities in rapidly rotating systems. J. Fluid Mech. 44(3), 441--460.
%
\item 
Cande, S.C., Kent, D.V., 1995. Revised calibration of the geomagnetic polarity timescale for the Late Cretaceous and Cenozoic, J. Geophys.\ Res.\ 100 (B4), 6093--6095. https://doi.org/10.1029/94JB03098.
%
\item
Christensen, U.R., Aubert, J., 2006. Scaling properties of convection-driven dynamos in rotating spherical shells and application to planetary fileds. Geophys.\ J. Int.\ 166, 97--114.
%
% \bibitem[\protect\citename{Coe and Glatzmaier, }2006] {Coe:2006}
\item
Coe, R.S., Glatzmaier, G.A., 2006. Symmetry and stability of the geomagnetic field. Geophys.\ Res.\ Lett. 33(21), doi:10.1029/2006GL027903.
%
\item
% {\color{red}
Davies, C.J., Constable, C.G., 2020. Rapid geomagnetic changes inferred from Earth observations and numerical simulations. Nat.\ Comm.\ 11, 3371. https://doi.org/10.1038/s41467-020-16888-0
% }
%
\item
Driscoll, P., Olson, P., 2009. Effects of buoyancy and rotation on the polarity reversal frequency of gravitationally driven numerical dynamos. Geophys.\ J. Int.\ 178 (3), 1337--1350. https://doi.org/10.111/j.1365-246X.2009.04234.x.
%
%\bibitem[\protect\citename{Glatzmaier et al., }1999] {Glatzmaier:1999}
\item
Glatzmaier, G.A., Coe, R.S., Hongre, L., Roberts, P.H., 1999. The role of the Earth's mantle in controlling the frequency of geomagnetic reversals. Nature 401(6756), 885--890.
%
%\bibitem[\protect\citename{Glatzmaier and Roberts, }1995] {GR:1995}
\item
Glatzmaier, G.A., Roberts, P.H., 1995. A three-dimensional self-consistent computer simulation of a geomagnetic field reversal. Nature 377(6546), 203--209.
%
\item
% {\color{teal}
Jones, C.A., Longbottom, A.W., Hollerbach, R., 1995. A self-consistent convection driven geodynamo model, using a mean field approximation. Phys.\ Earth Planet.\ Int.\ 92, 119--141.
% }
%
%\bibitem[\protect\citename{Kageyama and Sato, }1997] {Kageyama:1997}
\item
Kageyama, A., Sato, T., 1997. Generation mechanism of a dipole field by a magnetohydrodynamic dynamo. Phys.\ Rev.\ E 55(4), 4617--4626.
%
%\bibitem[\protect\citename{Kageyama et al., }1995] {Kageyama:1995}
\item
Kageyama, A., Sato, T., the Complexity Simulation Group, 1995. Computer simulation of a magnetohydrodynamic dynamo, II. Phys.\ Plasmas 2(5), 1421--1431.
%
%\bibitem[\protect\citename{Li et al., }2002] {Li:2002}
\item
Li, J., Sato, T., Kageyama, A., 2002. Repeated and sudden reversals of the dipole field generated by a spherical dynamo action. Science 295(5561), 1887--1890.
%
%\bibitem[\protect\citename{Liu and Olson, }2009] {Liu:2009}
% \item
% Liu, L., Olson, P., 2009. Geomagnetic dipole moment collapse by convective mixing in the core. Geophys.\ Res.\ Lett. 36(10), doi:10.1029/2009GL038130.
%
%\bibitem[\protect\citename{Matsui et al., }2014] {Matsui:2014}
\item
Matsui, H., King, E., Buffett, B., 2014. Multiscale convection in a geodynamo simulation with uniform heat flux along the outer boundary. Geochem.\ Geophys.\ Geosys. 15(8), 3212--3225. https://doi.org/10.1002/2014GC005432.
%
%\bibitem[\protect\citename{McFadden et al., }1991] {McFadden:1991}
% \item
% McFadden, P.L., Merrill, R.T., McElhinny, M.W., Lee, S., 1991. Reversals of the Earth's magnetic field and temporal variations of the dynamo families. J. Geophys.\ Res.\ 96(B3), 3923--3933.
%
\item
% {\color{teal}
Meduri, D.G., Biggin, A.J., Davies, C.J., Bono, R.K., Sprain, C.J., Wicht, J., 2021. Numerical dynamo simulations reproduce paleomagnetic field behavior. Geophys.\ Res.\ Lett.\ 48, e2020GL090544. https://doi.org/10.1029/2020GL90544.
% }
%
\item
% {\color{red}
Menu, M.D., Petitdemange, L., Galtier, S., 2020. Magnetic effects on fields morphologies and reversals in geodynamo simulations. Phys.\ Earth Planet.\ Inter.\ 307, 166542. https://doi.org/10.1016/j.pepi.2020.106542.
% }
%
\item
\sloppy
Nakagawa, T., Davies, C.J., 2022. Combined dynamical and morphological characterisation of geodynamo simulations. Earth Planet.\ Sci.\ Lett.\ 594, 117752. https://doi.org/10.1016/j.epsl.2022.117752. 
%
%\bibitem[\protect\citename{Nishikawa and Kusano, }2008] {Nishikawa:2008}
\item
Nishikawa, N., Kusano, K., 2008. Simulation study of the symmetry-breaking instability and the dipole field reversal in a rotating spherical shell dynamo. Phys.\ Plasmas 15(8), 082903.
%
%\bibitem[\protect\citename{Ogg et al., } 2005] {Ogg:2005}
% \item
% Ogg, J.G., Agterberg, F.P., Gradstein, F.M., 2005. The Cretaceous period. In Gradstein, F.M., Ogg, J.G., Smith, A.G., editors, A Geologic Time Scale 2004, pp.\ 344--383, Cambridge University Press, Cambridge.
%
\item 
Olson, P., Christensen, U.R., 2011. Dipole moment scaling for convection-driven planetary dynamos. Earth Planet.\ Sci.\ Lett.\ 250, 561--571.
%
%\bibitem[\protect\citename{Olson et al., }2011] {Olson:2011}
\item
\sloppy
Olson, P.L., Glatzmaier, G.A., Coe, R.S., 2011. Complex polarity reversals in a geodynamo model. Earth Planet.\ Sci.\ Lett.\ 304(1)--(2), 168--179, doi:10.1016/j.epsl.2011.01.031.
%
%\bibitem[\protect\citename{Sarson and Jones, }1999] {Sarson:1999}
\item
Sarson, G., Jones, C., 1999. A convection driven geodynamo reversal model. Phys. Earth Planet.\ Inter. 111, 3--20.
%
%\bibitem[\protect\citename{Sreenivasan et al., }2014] {Sreenivasan:2014}
%
\item 
% {\color{teal}
Sprain, C.J., Biggin, A.J., Davies, C.J., Bono, R.K., Meduri, D.G., 2019. An assessment of long duration geodynamo simulations using new paleomagnetic modeling criteria ($Q_{\rm PM}$). Earth Planet.\ Sci.\ Lett.\ 526, 115758. https://doi.org/10.1016/j.epsl.2019.115758.
% }
%
\item
Sreenivasan, B., Sahoo, S., Dhama, G., 2014. The role of buoyancy in polarity reversals of the geodynamo. Geophys.\ J. Int.\ 199(3), 1698--1709.
%
%\bibitem[\protect\citename{Takahashi et al., }2007] {TMH:2007}
\item
Takahashi, F., Matsushima, M., Honkura, Y., 2007. A numerical study on mangnetic polarity transition in an MHD dynamo model. Earth Planets Space 59(7), 665--673.
%
%\bibitem[\protect\citename{Tarduno et al., }2020] {Tarduno:2020}
\item
Tarduno, J.A., Cottrell, R.D., Bono, R.K., Oda, H., Davis, W.J., Fayek, M., van't Erve, O., Nimmo, F., Huang, W., Thern, E.R., Fearn, S., Mitra, G., Smirnov, A.V., Blackman, E.G., 2020. Paleomagnetism indicates that primary magnetite in zircon records a strong Hadean geodynamo. Proc.\ Nat.\ Acad.\ Sci. 117(5), 2309.
%
\item
% {\color{red}
Terra-Nova, F., Amit, H., 2024. Regionally-triggered geomagnetic reversals. Sci.\ Rep.\ 14, 9639. https://doi.org/10.1038/s41598-024-59849-z.
% }
%
%\bibitem[\protect\citename{Wicht and Olson, }2004] {Wicht:2004}
\item
Wicht, J., Olson, P., 2004. A detailed study of the polarity reversal mechanism in a numerical dynamo model: reversal mechanism. Geochem.\ Geophys.\ Geosys.\ 5 (3), doi:10.1029/2003GC000602.
%
%\bibitem[\protect\citename{Winch et al., }2005] {Winch:2005}
\item
Winch, D.E., Ivers, D.J., Turner, J.P.R., Stening, R. J., 2005. Geomagnetism and Schmidt quasi-normalization. Geophys.\ J. Int.\ 160 (2), 487--504.
%
\end{list}