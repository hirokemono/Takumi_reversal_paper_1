\documentclass[review]{elsarticle}
\usepackage{epstopdf}
\usepackage{amssymb} %maths
\usepackage{amsmath} %maths
\usepackage{timet,xcolor,graphicx}

\date{March 2022}

\input{mathdef}
\begin{document}

\title{Kinetic energy transfer during polarity reversal in numerical dynamo simulation}

\author[Tohoku]{Takumi Kera}
\ead{takumi.kera.r1@dc.tohoku.ac.jp}
\author[ucd]{Hiroaki Matsui\corref{cor1}}
\ead{hrmatsui@ucdavis.edu}
\author[TITECH]{Masaki Matsushima}
\ead{masaki.matsushima@eps.sci.titech.ac.jp}
\author[Tohoku]{Yuto Katoh}
\ead{yuto.katoh@tohoku.ac.jp}

\cortext[cor1]{Corresponding author}
\address[Tohoku]{Department of Geophysics, Tohoku University, Sendai, Japan.}
\address[ucd]{Department of the Earth and Planetary Sciences, University of California, Davis, CA, USA.}
\address[TITECH]{Department of Earth and Planetary Sciences, Tokyo Institute of Technology, Tokyo, Japan.}

\begin{keyword}
Geodynamo, Polarity reversals, Equatorial symmetry, Energy transfer
\end{keyword}

\begin{abstract}
The Earth has a magnetic field with a dominant dipole moment nearly parallel to the axis of Earth’s rotation. 
{\color{teal}
It is widely accepted that the geomagnetic field is sustained by fluid motion in the Earth’s outer core, the so-called dynamo action.
}
Paleomagnetic measurements have shown that the geomagnetic field has reversed its polarity many times. 
% It is widely accepted that the geomagnetic field is sustained by fluid motion in the Earth’s outer core, the so-called dynamo action. 
Some geodynamo simulations have been carried out to investigate the physical process of polarity reversals, and the equatorially antisymmetric flow during polarity reversals is found to be stronger than that during stable periods. 
On the other hand, convective motions in a rotating spherical shell have characteristics that the equatorially symmetric flow is dominant due to the effect of rotation. 
{\color{teal}
To investigate energy transfers between the equatorially symmetric and antisymmetric flows, we have performed geodynamo simulations with polarity reversals.
}
% We have performed a geodynamo simulation with polarity reversals, and analyzed energy transfers between the equatorially symmetric and antisymmetric flows.
The energy transfer to the equatorially antisymmetric flow is generally small.
{\color{teal}
Toward a polarity reversal, however, it increases in the following manner;
}
%The energy transfer to the equatorially antisymmetric flow is generally small, but it increases toward a polarity reversal as follows, 
(i) the rate of energy transfer from the equatorially symmetric flow to the magnetic field decreases, (ii) the rate of energy transfer from the equatorially symmetric flow to the antisymmetric flow by the advection increases, and (iii) the energy injection by the buoyancy force into the equatorially antisymmetric flow increases.
The present results suggest that the intense zonal flow caused by the intense upward flow inside the tangent cylinder in the either hemisphere can trigger a polarity reversal of the magnetic field.
\end{abstract}

\maketitle


\newpage
\section{Introduction}
\label{section:introduction}
the Earth has an intrinsic magnetic field which is dominated by a dipole component roughly aligned with the Earth's rotation axis. 
The amplitude and direction of the past geomagnetic field can be estimated by paleomagnetic observations using igneous or sedimentary rocks. Recent paleomagnetic observations have revealed that the geomagnetic field has been sustained for 4.2 billion years (Tarduno {\it et al.}, 2020 \cite{Tarduno:2020}).
In addition, the paleomagnetic observations and observations of the magnetic anomalies around the oceanic ridges have revealed that the direction of the geomagnetic field has frequently reversed on the geological time scales (Cande and Kent, 1995 \cite{Cande:1995}, for example). 
These results strongly support that the geomagnetic field is maintained by a flow motion of the liquid iron alloy in the Earth's outer core, so called the geodynamo process. 

% To understand the geodynamo processes, 
Numerical simulations {\color{red} of geodynamo} have 
% had large 
{\color{red} been playing an important} role to understand the geodynamo processes and flow dynamics of the Earth's outer core.
After Glatzmaier and Roberts (1995) \cite{GR:1995} and Kageyama et al.\ (1995) \cite{Kageyama:1995}, many magnetohydrodynamic (MHD) {\color{red} dynamo} simulations have been successfully performed to represent the characteristics of the geomagnetic field. 
The polarity reversal of the axial dipole component has also been represented in the geodynamo simulations ({\it e.g.} Glatzmaier and Roberts, 1995 \cite{GR:1995}; Sarson and Jones, 1999\cite{Sarson:1999}; Takahashi et al., 2007 \cite{TMH:2007}; Olson et al., 2011 \cite{Olson:2011}; Sreenivasan et al., 2014  \cite{Sreenivasan:2014}).

The convection in the outer core is likely to be dominated by the geostrophic balance, {\color{red} in which the Coriolis force is balancesd by the pressure gradient.} 
Busse (1970) \cite{Busse:1970} suggested that the convection in the rotating spherical shell occurs mainly outside the tangent cylinder, which is an imaginary cylinder with the radius of the inner core, and that the convection is characterized by multiple convective columns along with the rotation axis. 
The columnar helical flow in the anti-cyclonic convection columns generates the dipolar magnetic field due to the twisting magnetic field lines in the anti-cyclonic columns, and the zonal magnetic field line is extended with the cyclonic convection columns (Kageyama and Sato, 2017 \cite{Kageyama:97c}). 
These characteristics of the convection in the rotating spherical shell is symmetric with respect to the equatorial plane, and the Lorentz force is also symmetric with respect to the equatorial plane if the magnetic field only has the antisymmetric with respect to the equatorial plane such as the axial dipole field.
However, it is suggested that the equatorial symmetry of the convection in the outer core is broken and equatorially symmetric component of the magnetic field increases during the polarity reversals. 
Based on the paleomagnetic observations for the last 150 million years, the amplitude ratio of equatorially antisymmetric component of the non-axial dipole component to the total off-axis dipole component of the geomagnetic field is inversely correlated with the reversal occurrence (Coe and Glatzmaier, 2006 \cite{Coe:2006}). 
In numerical simulations, Glatzmaier {\it et al.} (1999) \cite{Glatzmaier:1999} showed that the equatorially antisymmetric components of the magnetic field are dominant when the axial dipole field is stably generated in their geodynamo simulations. 
The Earth-like magnetic dipole reversal is characterized by the stable dipole dominant magnetic field and spontaneous rapid reversal of the dipole component. 
Regarding flow motion, several numerical dynamo models with Earth-like dipole reversal represented a breakdown of the equatorial symmetry of the both meridional circulation and zonal flow  during the dipole reversal (Li {\it et al.}, 2002 \cite{Li:2002}; Wicht and Olson, 2004 \cite{Wicht:2004}).
Wicht and Olson (2004) \cite{Wicht:2004} concluded that the reversed axisymmetric toroidal electric current density ({\it i.e.} axisymmetric poloidal magnetic field) is generated near the outer boundary and tangent cylinder, which is an imaginary cylinder adjoint with the inner core boundary, by the plume with upwelling flow and that equatorially antisymmetric meridional circulation advects the reversed zonal current density to the whole outer core during the reversals.

The control factors for the dipole reversal have also been % studied by 
{\color{red} examined in} previous studies.
Glatzmaier et al.\ (1999) \cite{Glatzmaier:1999} performed thermally driven dynamo simulations with changing heat flux patterns at the outer boundary of the spherical shell and found that more reversals occurs in the case with smaller heat flux at high latitude ({\it i.e.} the outer core boundary).
Parameter regimes of the geodynamo simulations with Earth-like dipole reversal can generally be found between the parameter regimes to sustain stable intense dipolar field without reversal and that to generate weak and periodically changing dipole field with small scale magnetic field (Christensen and Aubert, 2006 \cite{aubert:2006}; Driscoll and Olson, 2009 \cite{driscoll:2009}).
Sreenivasan et al.\ (2014) \cite{Sreenivasan:2014} performed dynamo simulations with changing Rayleigh number and showed that occurrence of the dipole reversal increases with the Rayleigh number. 
Olson and Christensen (2006) \cite{Olson:2006} shows that the generated magnetic fields transit from dipole dominant field to multipolar magnetic field with increasing the Rayleigh number and Earth-like dipole reversal is represented the cases with the Rayleigh number which is transferring from the dipolar to multipolar regime. 
In addition, they also suggested that inertia can have a large role to drive the dipole reversals by scaling between dipole field strength and local Rossby number.
Nishikawa and Kusano (2008) \cite{Nishikawa:2008} focused on the energy transfer by the magnetic induction equation with splitting the equatorially symmetric and antisymmetric components. 
Nishikawa and Kusano (2008) suggests that the direction of the energy transfer changes between stable dipole period and during reversals. 
The energy of the symmetric components of the magnetic energy transfers to the antisymmetric components of the kinetic energy during the reversals, while the energy transfers from antisymmetric kinetic energy to the symmetric components of the magnetic energy in the stable dipole period.

In the present study, we focus on how the energies of equatorially symmetric and antisymmetric components of the flow are transferred by the buoyancy, advection, and Lorentz force. 
We perform MHD dynamo simulations in a rotating spherical shell modeled on the Earth's outer core and evaluate the work of the buoyancy, inertia term, and Lorentz force for the equatorially symmetric and antisymmetric components averaged over the spherical shell. 
% We obtained 12 reversals and investigate how energy fluxes changes around the dipole reversals. 
We will explain the models of the present dynamo simulation and describe the energy equations for the equatorially symmetric and antisymmetric components of the kinetic energy. 
In Section \ref{section:results}, we will show the results of the simulation and data analysis of the work of forces. 
In Section \ref{section:discussion}, we will discuss the comparison with the results of previous studies and magnetic field generation processes during the dipole reversal in the present simulation. 
Finally, conclusions will be described in Section \ref{section:discussion}.

\section{Method}
\label{section:method}

\subsection{Numerical method}

We perform numerical simulations of an MHD dynamo to investigate mechanism related to polarity reversals of the Earth's magnetic field.
The fluid outer core, in which dynamo action occurs, is represented by a spherical shell rapidly rotating with angular velocity $\bvec{\Omega} = \Omega \hat{\bvec{z}}$, where $\hat{\bvec{z}}$ is the unit vector aligned to the rotation axis.
The spherical shell with inner and outer radii, $r_i$ and $r_o$, respectively, and $r_i / r_o = 0.35$, is filled with an electrically conducting Boussinesq fluid, which is assumed to have constant kinetic viscosity $\nu$, thermal diffusivity $\kappa$, and magnetic diffusivity $\eta$.
This leads to an equation of continuity as
%
\begin{equation}
\nabla \cdot \bvec{u} = 0,
\label{eq:divu=0}
\end{equation}
%
where $\bvec{u}$ is the velocity field of core fluid.
The magnetic field, $\bvec{B}$, satisfies the Gauss's law given as
%
\begin{equation}
\nabla \cdot \bvec{B} = 0.
\label{eq:divB=0}
\end{equation}
%
The other nondimensional governing equations for the present MHD dynamo driven by thermally convective motions are derived as
%
\begin{equation}
\begin{array}{l}
\displaystyle
E \left( \frac{\partial\bvec{u}}{\partial t} +
 \bvec{\omega} \times \bvec{u} \right) =
 - \nabla \left( P + E\frac{1}{2} \bvec{u}^2 \right)
 + E \nabla^2 \bvec{u}
% \nonumber 
\\
\displaystyle
\hspace*{\fill}
 -2 \hat{\bvec{z}} \times \bvec{u}
 + Ra_f T \frac{\bvec{r}}{r_o}
 + \frac{1}{Pm} \bvec{J} \times \bvec{B} ,
\end{array}
\label{eq:momentum}
\end{equation}
%
\begin{equation}
\frac{\partial \bvec{B}}{\partial t} =
 \nabla \times (\bvec{u} \times \bvec{B} )
 + \frac{1}{Pm} \nabla^2 \bvec{B} ,
\label{eq:induction}
\end{equation}
%
\begin{equation}
\frac{\partial T}{\partial t} 
 + ( \bvec{u} \cdot \nabla ) T =
 \frac{1}{Pr} \nabla^2 T ,
\label{eq:heat}
\end{equation}
%
where $t$ is the time, $\bvec{\omega}$ is the vorticity, $P$ is the pressure, $\bvec{r}$ is the position vector, and $\bvec{J}$ is the electric current density.
The length, time, pressure, temperature, and magnetic field are respectively scaled by $D$, $D^2/\nu$, $\nu^2 /D^2$, $\beta_o D$, and $(\rho \mu_0 \Omega \eta )^{1/2}$,
where $D = r_o - r_i$, $\rho$, $\mu_0$, and $\beta_o$ are the thickness of the outer core, the mean density of the {\color{teal}electrically} conductive fluid, the magnetic permeability of vacuum, and the temperature gradient at the core-mantle boundary (CMB), respectively.
The dimensionless numbers included in the governing equations are the Rayleigh number, $Ra_f$, the Ekman number, $E$, the Prandtl number, $Pr$, and the magnetic Prandtl number, $Pm$, which are defined as
%
\begin{equation}
% Ra_f = \frac{\alpha g_o \beta_o r_{o}^{2}}{\nu \Omega},~~
{\color{teal}Ra_f = \frac{\alpha g_o \beta_o D^{2}}{\nu \Omega},}~~
E = \frac{\nu}{\Omega D^2},~~
Pr = \frac{\nu}{\kappa},~~ \mbox{and \ }
Pm = \frac{\nu}{\eta},
\label{eq:dimensionless_numbers}
\end{equation}
%
where $\alpha$ is the coefficient of thermal volume expansion, and $g_o$ is the amplitude of gravity at $r = r_o$.

The solenoidal vector fields, $\bvec{u}$ and $\bvec{B}$, are decomposed into the toroidal and poloidal components as
%
\begin{equation}
    \bvec{u} (\bvec{r}, t) = \nabla \times ( u_T (\bvec{r}, t) \hat{\bvec{r}} ) + \nabla \times \nabla \times ( u_S (\bvec{r}, t) \hat{\bvec{r}} ),
\label{eq:uT_uS}
\end{equation}
%
\begin{equation}
    \bvec{B} (\bvec{r}, t) = \nabla \times ( B_T (\bvec{r}, t) \hat{\bvec{r}} ) + \nabla \times \nabla \times ( B_S (\bvec{r}, t) \hat{\bvec{r}} ),
\label{eq:BT_BS}
\end{equation}
%
where $\hat{\bvec{r}}$ is the radial unit vector.
Toroidal and poloidal scalar functions for the velocity and magnetic fields, $u_T (\bvec{r}, t)$, $u_S (\bvec{r}, t)$, $B_T (\bvec{r}, t)$, and $B_S (\bvec{r}, t)$, 
are expanded into spherical harmonics in the horizontal directions. For example, $u_{S} (\bvec{r}, t)$ is expanded as
%
\begin{equation}
    u_{S} (\bvec{r}, t) = \sum_{l=1}^{L_{\rm max}} \sum_{m=-l}^{l} u_{Sl}^{\ m} (r, t) Y_l^{|m|} (\theta, \phi),
\label{eq:Us_expansion}
\end{equation}
%
where $L_{\rm max}$ is the truncation of spherical harmonic expansion, and
%
\begin{equation}
Y_l^{|m|} (\theta, \phi) = \left\{
 \begin{array}{ll}
 P_l^m(\cos\theta)\cos m\phi & (m = 0, 1, 2, \cdots, l)
 \\
 P_l^{|m|}(\cos\theta)\sin |m|\phi & (m = -1, -2, \cdots, -l) ,
 \end{array}
\right.
\label{eq:def_of_Ylm}
\end{equation}
%
where, $P_l^m (\cos \theta )$ is a Schmidt quasi-normalized associated Legendre polynomial with degree $l$ and order $m$ (Winch et al., 2005). 
% \cite{Winch:2005}).
The temperature field is also expanded into the spherical harmonics coefficients as 
%
\begin{equation}
    T(\bvec{r}, t) = \sum_{l=0}^{L_{\rm max}} \sum_{m=-l}^{l} T_l^m (r, t) Y_l^{|m|} (\theta, \phi).
\label{eq:T_expansion}
\end{equation}
%

In the radial direction, scalar functions, $u_T (\bvec{r}, t)$, $u_S (\bvec{r}, t)$, $B_T (\bvec{r}, t)$, $B_S (\bvec{r}, t)$, and $T(\bvec{r}, t)$, are differentiated by the second-order finite difference method. 
To obtain high spatial resolution near the boundaries and a smooth transition of grid spacing, the so-called Chebyshev grid is used for the radial grid points defined as
%
\begin{equation}
r_n = r_i + \frac{r_o - r_i}{2} \left\{ 1 - \cos \left( \pi \frac{n-1}{N_r-1} \right) \right\} ~~\;\;\;\; (n = 1, \cdots , N_r) ,
\label{eq:def_of_rn}
\end{equation}
%
where $N_r$ is the number of radial grids.

For the time integration, the Crank-Nicolson scheme is adopted for the diffusion terms, and the second order Adams-Bashforth scheme is used for the other terms.

The no-slip condition for the velocity field is imposed at impermeable boundary surfaces, and the inner core is assumed to co-rotate with the mantle, which leads to $\bvec{u} = {\bf 0}$.
The regions outside the spherical shell corresponding to the inner core and the mantle are assumed to be electrical insulators for simplicity, so that the magnetic field in the spherical shell is continuous to a potential field both at $r = r_i$ and $r = r_o$.
%An 
{\color{red}
A
}
uniform temperature gradient at the CMB is imposed as $-\partial T_0^0 / \partial r |_{r = r_o} = r_{o}^{-2} = 0.4225$. 
To satisfy $T_0^0 (r, t)$ with $\nabla^2 T_0^0 = 0$,
the temperature gradient at the inner core boundary (ICB) is determined from the balance between the heat flux into at $r = r_i$ and out of $r = r_o$. 
Consequently, the uniform temperature gradient at the ICB is set to $-\partial T_0^0 / \partial r |_{r = r_i} = r_i^{-2}$.

The initial condition for numerical simulations is set as follows; $\bvec{u} = {\bf 0}$ for the velocity field, a geocentric magnetic dipole moment whose tilt angle from the rotation axis is $\pi / 4$, and a component of degree 4 and order 4 of spherical harmonics for the temperature as adopted by a dynamo benchmark (Christensen et al., 2001).
%The dimensionless parameters are set as $Ra_f = 2000$, $E = 6 \times 10^{-4}$, $Pr = 1$, and $Pm = 5$ on the basis of results by Sreenivasan et al.\ (2014). % \cite{Sreenivasan:2014}.
{\color{teal}
The dimensionless parameters are set as $Ra_f = 2000$, $E = 6.0 \times 10^{-4}$, $Pr = 1.0$, and $Pm = 5.0$ on the basis of results by Sreenivasan et al.\ (2014).
}

\subsection{Equatorial symmetry}

We investigate possible variations in equatorial symmetry of the velocity and magnetic fields in relation to polarity reversals.
Dimensionless kinetic energy density and dimensionless magnetic energy density are respectively defined as
%
\begin{equation}
E_{\rm kin} = \frac{1}{V}
  \int_V \frac{1}{2} \bvec{u}^2 d V ,
\label{eq:Ekin}
\end{equation}
%
\begin{equation}
E_{\rm mag} = \frac{1}{V Pm E}
  \int_V \frac{1}{2} \bvec{B}^2 d V ,
\label{eq:Emag}
\end{equation}
%
where integrals are carried out over the volume of spherical shell, $V$.
Temporal variations of kinetic and magnetic energy densities can be derived from (\ref{eq:momentum}) and (\ref{eq:induction}), respectively.
Their energy equations are given as
%
\begin{eqnarray}
\frac{\partial}{\partial t}
 \int_V \frac{|\bvec{u}|^2}{2} d V
 &=& \int_V \left\{
    - |\bvec{\omega}|^2 
    + \frac{Ra_f}{E}T \bvec{u}\cdot 
    \frac{\bvec{r}}{r_o}
      \right.
\nonumber \\
 & & \left.
    - \bvec{u} \cdot (\bvec{\omega}\times\bvec{u})
    + \frac{1}{Pm E} \bvec{u} \cdot
                     (\bvec{J} \times \bvec{B})
      \right\} d V,
\label{eq:energy_u}
\end{eqnarray}
%
\begin{eqnarray}
\frac{1}{Pm E}\frac{\partial}{\partial t}
 \int_V \frac{|\bvec{B}|^2}{2} d V
  & = & 
- \frac{1}{Pm E}\int_V \left\{
      \bvec{u} \cdot ( \bvec{J} \times \bvec{B} )
\right. \nonumber \\
 & &    \left.
    + \frac{1}{Pm} |\bvec{J}|^2
    + \nabla \cdot ( \bvec{E} \times \bvec{B} ) 
      \right\} d V.
\label{eq:energy_B}
\end{eqnarray}
%
Any vector can be divided into equatorially symmetric and antisymmetric constituents.
{\color{teal}
For example, 
}
we represent the velocity and magnetic fields as
%
\begin{equation}
\bvec{u} = \bvec{u}^s + \bvec{u}^a, ~~~
\bvec{B} = \bvec{B}^s + \bvec{B}^a,
\label{eq:eqas}
\end{equation}
%
where superscripts $s$ and $a$ denote equatorially symmetric and antisymmetric fields, respectively.
We then derive the energy equations for the equatorially symmetric and antisymmetric velocity field, respectively, given as
%
\begin{eqnarray}
\displaystyle
\frac{\partial}{\partial t}
 \int \frac{|\bvec{u}^s|^2}{2} d V 
 &=& \int \left\{
      \frac{Ra_f}{E}T^s \bvec{u}^s\cdot \frac{\bvec{r}}{r_{o}}
     \right.
\nonumber \\
& & \displaystyle
\hspace*{2em}
     \left.
    + \frac{1}{Pm E} \bvec{u}^s \cdot
                (\bvec{J}^s \times \bvec{B}^a
                +\bvec{J}^a \times \bvec{B}^s)
      \right.
\nonumber \\
& &\displaystyle
\hspace*{3em}
      \left.
    - \bvec{u}^s \cdot 
       (\bvec{\omega}^s \times \bvec{u}^a)
    - |\bvec{\omega}^a|^2 
      \right\} d V,
\label{eq:energy_us}
\end{eqnarray}
%
\begin{eqnarray}
\displaystyle
\frac{\partial}{\partial t}
 \int \frac{|\bvec{u}^a|^2}{2} d V 
 & = & \displaystyle
% \hspace*{1em}
\int \left\{
      \frac{Ra_f}{E}T^a \bvec{u}^a\cdot \frac{\bvec{r}}{r_{o}}
     \right.
\nonumber \\
& & \displaystyle
\hspace*{2em}
     \left.
    + \frac{1}{Pm E} \bvec{u}^a \cdot
                (\bvec{J}^a \times \bvec{B}^a
                +\bvec{J}^s \times \bvec{B}^s)
      \right.
\nonumber \\
& & \displaystyle
\hspace*{3em}
      \left.
     {\color{red} - } \bvec{u}^a \cdot 
       (\bvec{\omega}^s \times \bvec{u}^s)
    - |\bvec{\omega}^s|^2 
      \right\} d V.
\label{eq:energy_ua}
\end{eqnarray}
%
In the same way, the energy equations for the equatorially symmetric and antisymmetric magnetic field are respectively obtained as
%
\begin{eqnarray}
\displaystyle
\frac{1}{Pm E}\frac{\partial}{\partial t}
 \int \frac{|\bvec{B}^s|^2}{2} d V
 & = & - \frac{1}{Pm E}\int \left\{
      \bvec{u}^a \cdot 
          ( \bvec{J}^a \times \bvec{B}^a )
    + \bvec{u}^s \cdot 
          ( \bvec{J}^a \times \bvec{B}^s )
    \right.
\nonumber \\
\hspace{3em}
&& \displaystyle
    \left.
    + \frac{1}{Pm} |\bvec{J}^a|^2
    + \nabla \cdot ( \bvec{E}^a \times \bvec{B}^s ) 
      \right\} d V,
% \end{array}
\label{eq:energy_Bs}
\end{eqnarray}
%
and 
%
\begin{eqnarray}
\displaystyle
\frac{1}{Pm E}\frac{\partial}{\partial t}
 \int \frac{|\bvec{B}^a|^2}{2} d V 
& = & \displaystyle
- \frac{1}{Pm E}\int \left\{
      \bvec{u}^a \cdot 
          ( \bvec{J}^s \times \bvec{B}^s )
    + \bvec{u}^s \cdot ( \bvec{J}^s \times \bvec{B}^a )
    \right.
\nonumber \\
\hspace{3em}
 & & \displaystyle
    \left.
    + \frac{1}{Pm} |\bvec{J}^s|^2
    + \nabla \cdot ( \bvec{E}^s \times \bvec{B}^a ) 
      \right\} d V.
\label{eq:energy_Ba}
\end{eqnarray}

%The right-hand-sides of (\ref{eq:energy_us}) and (\ref{eq:energy_ua}) show energy transfer due to respective forces as follows.
%The first terms corresponding to the work by buoyancy mean that equatorially symmetric and antisymmetric temperature fields contribute to kinetic energy for the equatorially symmetric and antisymmetric velocity fields, respectively.
%{\color{blue}
%The second terms are the work by the Lorentz force which show energy transfer between kinetic and magnetic energies. The energy transfers by the Lorentz force from the symmetric component of the kinetic energy to the antisymmetric and symmetric components of the magnetic energy are described by $- \bvec{u}^s \cdot (\bvec{J}^s \times \bvec{B}^a)$ and $- \bvec{u}^s \cdot (\bvec{J}^a \times \bvec{B}^s)$ in (\ref{eq:energy_us}), respectively. 
%From the antisymmetric component of the kinetic energy, the energy transfers by the Lorentz force from the kinetic energy to the antisymmetric and symmetric components of the magnetic energy are described by $- \bvec{u}^a \cdot (\bvec{J}^s \times \bvec{B}^s)$ and $- \bvec{u}^a \cdot (\bvec{J}^a \times \bvec{B}^a)$ in (\ref{eq:energy_ua}), respectively. 
%Consequently, the $- \bvec{u}^s \cdot (\bvec{J}^a \times \bvec{B}^s)$ and $- \bvec{u}^a \cdot (\bvec{J}^a \times \bvec{B}^a)$ appear in (\ref{eq:energy_Bs}) as the energy input to the symmetric component of the magnetic field by the magnetic induction, and $- \bvec{u}^s \cdot (\bvec{J}^s \times \bvec{B}^a)$ and $- \bvec{u}^a \cdot (\bvec{J}^s \times \bvec{B}^s)$ appear in (\ref{eq:energy_Ba}) as the energy input to the antisymmetric component of the magnetic field by the magnetic induction.

%The work of inertia term $-\bvec{u} \cdot \left(\bvec{\omega} \times \bvec{u} \right)$ in equation (\ref{eq:energy_u}) is also expanded into that for symmetric and ant-symmetric component in equations (\ref{eq:energy_us}) and (\ref{eq:energy_ua}) same as the work of Lorentz force. However, the terms $- \bvec{u}^s \cdot (\bvec{\omega}^a \times \bvec{u}^s) = - \bvec{\omega}^a \cdot (\bvec{u}^s \times \bvec{u}^s) $ and $- \bvec{u}^a \cdot (\bvec{\omega}^a \times \bvec{u}^a) = - \bvec{\omega}^a \cdot (\bvec{u}^a \times \bvec{u}^a)$ are equal to be zero. The remaining terms $-\bvec{u}^s \cdot (\bvec{\omega}^s \times \bvec{u}^a)$ in equation (\ref{eq:energy_us}) and $- \bvec{u}^a \cdot (\bvec{\omega}^s \times \bvec{u}^s)$ in equation (\ref{eq:energy_ua}) represent the kinetic energy transfer between the symmetric and anti-symmetric components by inertia because $- \bvec{u}^a \cdot (\bvec{\omega}^s \times \bvec{u}^s)$ in equation (\ref{eq:energy_ua}) is equal to be $\bvec{u}^s \cdot (\bvec{\omega}^s \times \bvec{u}^a)$. Consequently, the inertia term does not contribute to total kinetic energy.
%}
%The second terms, the work by the Lorentz force, show energy transfer between kinetic and magnetic energies, as they are also found in (\ref{eq:energy_Bs}) and (\ref{eq:energy_Ba}); 
%that is, $\bvec{u}^s \cdot (\bvec{J}^s \times \bvec{B}^a)$ corresponds to energy transfer between $\bvec{u}^s$ and $\bvec{B}^a$, $\bvec{u}^s \cdot (\bvec{J}^a \times \bvec{B}^s)$ to that between $\bvec{u}^s$ and $\bvec{B}^s$, $\bvec{u}^a \cdot (\bvec{J}^a \times \bvec{B}^a)$ to that between $\bvec{u}^a$ and $\bvec{B}^a$, and $\bvec{u}^a \cdot (\bvec{J}^s \times \bvec{B}^s)$ to that between $\bvec{u}^a$ and $\bvec{B}^s$.
%In other words, $-\bvec{u}^s \cdot (\bvec{J}^s \times \bvec{B}^a)$ contributes to temporal variations of $\bvec{B}^a$ caused by $\bvec{u}^s$, 
%$-\bvec{u}^s \cdot (\bvec{J}^a \times \bvec{B}^s)$ to those of $\bvec{B}^s$ by $\bvec{u}^s$,
%$-\bvec{u}^a \cdot (\bvec{J}^a \times \bvec{B}^a)$ to those of $\bvec{B}^a$ by $\bvec{u}^a$, and 
%$-\bvec{u}^a \cdot (\bvec{J}^s \times \bvec{B}^s)$ to those of $\bvec{B}^s$ by $\bvec{u}^a$, as found in (\ref{eq:energy_Bs}) and (\ref{eq:energy_Ba}).
% The third terms in the right-hand-sides of (\ref{eq:energy_us}) and (\ref{eq:energy_ua}) express energy transfer between $\bvec{u}^s$ and $\bvec{u}^a$ due to the advection, which does not contribute to total kinetic energy.
{\color{teal}
The right-hand-sides (RHS) of (\ref{eq:energy_us})-- (\ref{eq:energy_Ba}) show the energy transfer due to respective factors, and they are summarized as follows.
\begin{enumerate}
\item 
The first terms in the RHS of (\ref{eq:energy_us}) and (\ref{eq:energy_ua}) correspond to the work by buoyancy, and they mean that equatorially symmetric and antisymmetric temperature fields, $T^s$ and $T^a$, contribute to kinetic energy for the equatorially symmetric and antisymmetric velocity fields, $\bvec{u}^s$ and $\bvec{u}^a$, respectively.

\item 
The second terms in the RHS of (\ref{eq:energy_us}) and (\ref{eq:energy_ua}) correspond to the work by the Lorentz force, and they show energy transfer between kinetic and magnetic energies.
The terms, $+\bvec{u}^s \cdot (\bvec{J}^s \times \bvec{B}^a)$ and $+\bvec{u}^s \cdot (\bvec{J}^a \times \bvec{B}^s)$, show the energy transfer from the kinetic energy for the equatorially symmetric flow, $\bvec{u}^s$, to the magnetic energy for the equatorially antisymmetric and symmetric magnetic fields, $\bvec{B}^a$  and $\bvec{B}^s$, respectively. 
Consequently, $-\bvec{u}^s \cdot (\bvec{J}^s \times \bvec{B}^a)$ in (\ref{eq:energy_Ba}) and $-\bvec{u}^s \cdot (\bvec{J}^a \times \bvec{B}^s)$ in (\ref{eq:energy_Bs}) appear as the energy transfer to the magnetic energy for $\bvec{B}^a$ and $\bvec{B}^s$ by the magnetic induction, respectively.
In the same way, the terms, $+\bvec{u}^a \cdot (\bvec{J}^s \times \bvec{B}^s)$ and $+\bvec{u}^a \cdot (\bvec{J}^a \times \bvec{B}^a)$, show the energy transfer from the kinetic energy for $\bvec{u}^a$ to the magnetic energy for $\bvec{B}^s$ and $\bvec{B}^a$ by the magnetic induction, respectively.
Therefore, $-\bvec{u}^a \cdot (\bvec{J}^s \times \bvec{B}^s)$ in (\ref{eq:energy_Ba}) and $-\bvec{u}^a \cdot (\bvec{J}^a \times \bvec{B}^a)$ in (\ref{eq:energy_Bs}) appear as the energy transfer to the magnetic energy for $\bvec{B}^a$ and $\bvec{B}^s$ by the magnetic induction, respectively.

\item 
The third terms in the RHS of (\ref{eq:energy_us}) and (\ref{eq:energy_ua}) correspond to the work by inertia, and they do not contribute to total kinetic energy.
The term, $-\bvec{u} \cdot (\bvec{\omega} \times \bvec{u})$, in (\ref{eq:energy_u}) is expressed in terms of $\bvec{u}^s$ and $\bvec{u}^a$, but $-\bvec{u}^s \cdot (\bvec{\omega}^a \times \bvec{u}^s) = -\bvec{\omega}^a \cdot (\bvec{u}^s \times \bvec{u}^s)$ and $-\bvec{u}^a \cdot (\bvec{\omega}^a \times \bvec{u}^a) = -\bvec{\omega}^a \cdot (\bvec{u}^a \times \bvec{u}^a)$ vanish. 
The remaining terms, $-\bvec{u}^s \cdot \bvec{\omega}^a \times \bvec{u}^a$ in (\ref{eq:energy_us}) and $-\bvec{u}^a \cdot (\bvec{\omega}^s \times \bvec{u}^a)$ in (\ref{eq:energy_ua}), represent the energy transfer between the kinetic energy for $\bvec{u}^s$ and $\bvec{u}^a$ due to inertia as found by $-\bvec{u}^a \cdot (\bvec{\omega}^s \times \bvec{u}^s) = \bvec{u}^s \cdot (\bvec{\omega}^s \times \bvec{u}^a)$.

\item 
The fourth terms in the RHS of (\ref{eq:energy_us}) and (\ref{eq:energy_ua}) indicate the viscous dissipation.

\item 
The third terms in the RHS of (\ref{eq:energy_Bs}) and (\ref{eq:energy_Ba}) indicate the Ohmic dissipation.

\item 
The fourth terms in the RHS of (\ref{eq:energy_Bs}) and (\ref{eq:energy_Ba}) correspond to the Poynting flux which can be expressed, for example, as
\begin{equation}
-\int \nabla \cdot
  ( \bvec{E}^a \times \bvec{B}^s ) d V =
 \oint (\bvec{E}^a \times \bvec{B}^s )
   \cdot \hat{\bvec{r}} d S .
\label{eq:Poynting}
\end{equation}
\end{enumerate}
}
%The fourth terms indicate the viscous dissipation. 
%The third terms in the right-hand-sides of (\ref{eq:energy_Bs}) and (\ref{eq:energy_Ba}) indicate the Ohmic dissipation.
%The forth terms correspond to the Poynting flux which can be expressed, for example, as
%
%\begin{equation}
%-\int \nabla \cdot
%  ( \bvec{E}^a \times \bvec{B}^s ) d V =
% \oint (\bvec{E}^a \times \bvec{B}^s )
%   \cdot \hat{\bvec{r}} d S .
%\label{eq:Poynting}
%\end{equation}
%

We implemented subroutines to evaluate these volume averaged energy transfers decomposed to the equatorial symmetry into a numerical dynamo code, Calypso Ver.~1.2 (Matsui et al., 2014) 
% \cite{Matsui:2014}) 
and performed dynamo simulations with polarity reversals.
%
The source code and documents of Calypso Ver.~1.2 can be found in the following URL;\\
https://github.com/geodynamics/calypso.\\



\section{Results}
\label{section:results}

We performed a dynamo simulation with $E = 6.0 \times 10^{-4}$, $Ra_f = 2000$, $Pr = 1.0$, and $Pm = 5.0$. 
The simulation is performed approximately 80 times %of 
% the magnetic diffusion time 
{\color{red}
$\tau_{\eta} = L^{2}/ \eta$ (the magnetic diffusion time), 
}
and the average ratio of magnetic to kinetic energies are $E_{\rm mag} / E_{\rm kin}$ is approximately 0.63. 
As seen the evolution of dipole tilt angle in Fig.~\ref{fig:sph_shell_275_full}, the present case has stable dipole period and periods with 13 dipole reversal and 6 excursion events. 
In the present study, we choose 9 periods which are shown by arrows in Fig.~\ref{fig:sph_shell_275_full} including these reversals and investigate energy transfer among the equatorially symmetric and anti-symmetric components of the kinetic and magnetic energies. 
We also run 7 simulations  starting from snapshots in period 1, 2, and 3. 
As seen in Fig.~\ref{fig:dipole_tilt_retries}, these results departs from the initial simulation result except for the case Retry 3-2, but these retried cases also have reversals and excursions. 
We will discuss the reason why these retried cases go to different solutions in the discussion section.

\subsection{Characteristics of the field during polarity reversals}

First, we investigate characteristics of the temperature field during a polarity reversal by choosing a result between $13.0 < t/\tau_{\eta} < 14.0$. 
As shown in the panel (b) of Fig.~\ref{fig:temperature_rendering}, equatorially antisymmetric component of the temperature is larger than the equatorially symmetric component of the temperature during the reversals and excursion, while the equatorially symmetric component of the temperature is approximately 1.5 times larger than the equatorially antisymmetric component. 
Looking at the three dimensional structure of the temperature, hot material rises strongly in the southern hemisphere during the reversal, while hot region inside the tangent cylinder can be observed in the both hemisphere (See Fig.~\ref{fig:temperature_rendering} and movie in the supplement materials). 
However, looking at the radial magnetic fields at the outer boundary of the spherical shell, any corresponding feature to the temperature field is not observed.

In the following investigation, we will take time averages of the fields during the stable dipole and reversing periods. 
We choose the amplitude of the axial dipole component of the Gauss coefficients $g_{1}^{0}$ at $r = 2.8$ obtained bt the poloidal magnetic field at the outer boundary of the shell $r = r_{o}$. 
As shown in Fig.~\ref{Fig:Reversal_period_def}, we define the reversing periods when $\left( g_{1}^{0} \right)^2 < 4.9 \times 10^{-5}$ in the present study. 

As shown in Fig.~\ref{fig:temperature_rendering}, large warm region is only observed in the southern hemisphere. 
We compare zonally power spectra of the equatorially symmetric and antisymmetric components of the kinetic energy and temperature in the stable and reversal periods using the Retry 1-1 case. 
The changes between stable and reversal periods are quite different between the axisymmetric ($m = 0$) components and non-axisymmetric components ($m \ne 0$). 
In the non-axisymmetric components, equitorially symmetric component of kinetic energy and temperature is always larger than the equatorially antisymmetric component, and there is no significant changes between the stable and reversing dipole period. 
The toroidal kinetic energy at $m > 10$ in the reversing periods increases from the stable period, but the difference is not significant. 
On the other hand, significant changes can be observed in the axisymmetric component of the temperature and toroidal kinetic energy. 
In the toroidal kinetic energy, equatorially antisymmetric component in the reversing period is approximately twice of that in the stable dipole period, while there is almost no changes in the amplitude of the equatorially symmetric component of the axisymmetric kinetic energy.
Consequently, the equatorially antisymmetric component of the toroidal kinetic energy is larger than the equatorially symmetric component of them. 
The axisymmetric component of temperature in the reversing period increases approximately 3 times of that in the stable periods. 
The equatorially antisymmetric component of the axisymmetric temperature becomes larger than the non-axisymmetric components of the temperature, while the non-axisymmetric components of the kinetic energy is larger than the equatorially antisymmetric components of the kinetic energy. 
Consequently, the equatorially antisymmetric component of the temperature is larger than the equatorially symmetric components during the reversing in Fig.~\ref{fig:temperature_rendering}. 
In addition, the equatorially antisymmetric component of the axisymmetric temperature is larger than the equatorially symmetric component of that even in the stable dipole period.

% \subsection{Investigation of energy flows}
{\color{red}
\subsection{Investigation of energy transfers}
}

We investigate the energy 
% fluxes 
{\color{red} transfer}
during the dipole reversal with splitting the contribution of the equatorially symmetric and anti-symmetric components of the flow, magnetic field, and temperature as described in the equations (\ref{eq:energy_us}) and (\ref{eq:energy_ua}).

First, we investigate time evolution of the energy fluxes for one reversal in Retry 1-1 case. 
The time evolutions of the energy fluxes for the equatorially symmetric and antisymmetric kinetic energies are plotted included the evolution of dipole tilt angle in Fig.~\ref{fig:energy_flux_evolution_retry1_1}. 
As described in the previous subsection, the equatorially antisymmetric component of kinetic energy does not overcome the equatorially symmetric component of the kinetic energy. 
In addition, there is no significant changes in the overall amplitude of energy 
% fluxes 
{\color{red} transfers}
during the stable and reversing dipole periods. 
{\color{red} Energy transfer by}
the buoyancy flux is the largest energy input for the equatorially symmetric and antisymmetric components of the kinetic energy. 
The work of Lorentz force is always negative, %value, 
which shows that the both components of kinetic energy are transferred to the magnetic energy. 
The energy 
% flux 
{\color{red} transfer}
for the advection term is the smallest amplitude in these energy 
% fluxes, 
{\color{red} transfers,}
but the advection term always transfers energy from the equatorially symmetric component to antisymmetric component 
% (see panels (c) and (e)). 
{\color{red}
(see Fig.~\ref{fig:energy_flux_evolution_retry1_1}c and e).
}
To investigate the difference between stable and reversing dipole periods, we plot the perturbation from the time average over the period for Retry 1-1 
% in the panels (d) and (f) 
{\color{red}
in Fig.~\ref{fig:energy_flux_evolution_retry1_1}d and f.
} 
In this period, there is one reversal event at $t = 33.0$, and one excursion at $t = 36.2$. 
The same behavior can be found in the both events. 
For the equatorially symmetric components, the most significant change is the increasing the perturbation of the work of Lorentz force. 
This change corresponds to the decreasing the negative energy flux to kinetic energy, {\it i.e.}, decreasing the energy transfer to the magnetic energy due to the decreasing the magnetic energy. 
On the other hand, the change of the work of the Lorentz force for the equatorially antisymmetric components of the kinetic energy is less significant than that for the equatorially symmetric kinetic energy. 
The increase of the buoyancy flux and work by the advection are more significant than the work of Lorentz force. 
Looking at more detail, the work of the advection starts increasing first, and the buoyancy flux in the next. 
The energy transfer by inertia increases the equatorially antisymmetric component of the axisymmetric toroidal flow, because the buoyancy flux can only be the energy input of the poloidal flow.

We also investigate the energy 
% flux 
{\color{red} transfer}
for the kinetic energy for the 11 periods in total including the stable and reversing periods. 
We took a time average of the work of Lorentz force, advection, and buoyancy flux for the equatorially symmetric and anti-symmetric components of the flow in the reversing period, and take a perturbation from the time averages of these terms. 
As seen in Fig.~\ref{Fig:Change_flux_summary_6grp}, the perturbation of the work of Lorentz force and advection has similar behavior in the all periods, while the perturbation of the buoyancy flux has large variation among the periods.
For the equatorially symmetric flow components, energy transfer from equatorially symmetric component of flow to the magnetic field decreases ({\it i.e.} perturbation of work of Lorentz force increases). 
For the equatorially antisymmetric components of the flow, the energy transfer by the work of the Lorentz force also decreases, but the amplitude is small. 
The largest change is the energy transfer from equitorially symmetric flow to the equatorially antisymmetric flow by the work of advection. 
The work of buoyancy by the equatorially antisymmetric component of the temperature also increases in the most of cases, but the amplitude is still smaller than that by the advection. 
Considering the temperature structure during the reversal in Fig.~\ref{fig:temperature_rendering} and change of the power spectra of the kinetic energy in Fig.~\ref{fig:KE_temp_spectra_m}, the intense axisymmetric and equatorially antisymmetric flow is induced by the advection to sustain the thermal wind inside the tangent cylinder. 

Taking into account the order of changing the energy 
% fluxes 
{\color{red} transfers}
in Fig.~\ref{fig:energy_flux_evolution_retry1_1}, we can summarize the process of energy transfer during the reversal as shown in Fig.~\ref{Fig:schematic_reversal}. 
First, the energy transfer to the magnetic energy decreases. 
Then, the advection transfers from the equatorially symmetric flow to axisymmetric zonal flow inside the tangent cylinder of the either hemisphere to sustain the thermal wind balance. 
And buoyancy flux inside the tangent cylinder also drives upwelling flow and enhance the equatorially antisymmetric temperature patterns.

%
\begin{figure}[ht]
\begin{center}
\[
\begin{array}{c}
\includegraphics*[width=120mm]{Figures/whole_energies.pdf} \\
\includegraphics*[width=120mm]{Figures/whole_dipole_angle.pdf}
\end{array}
\]
\end{center}
\caption{
Time evolution of kinetic and magnetic energies (top panel) and the dipole tile angle (bottom panel) throughout the present simulation
{\color{red}
(approximately 80 times of the magnetic diffusion times).
} 
The dipole tilt angle between $16.2 < t / \tau_{\eta} < 26.9$ is not plotted due to missing of the data. 
{\color{red}
% We performed the simulation to approximately 80 times of the magnetic diffusion times and obtained 12 reversals after $t = 5.0 \tau_{\eta}$. 
After $t = 5.0 \tau_{\eta}$, 12 polarity reversals occurred.
% Range of time averaging is shown by arrows on the bottom panel.
Double headed arrows in the bottom panel show eight periods during which data analysis is carried out.
}
}
\label{fig:sph_shell_275_full}
\end{figure}
%
%
\begin{figure}[ht]
\begin{center}
\[
\begin{array}{cc}
\mbox{Period 1} & \mbox{Period 2} \\
\includegraphics*[width=60mm]{Figures/dipole_angle_retry_1.pdf} &
\includegraphics*[width=60mm]{Figures/dipole_angle_retry_2.pdf} \\
\multicolumn{2}{c}{\mbox{Period 3}} \\
\includegraphics*[width=60mm]{Figures/dipole_angle_retry_33.pdf} &
\includegraphics*[width=60mm]{Figures/dipole_angle_retry_3.pdf}
\end{array}
\]
\end{center}
\caption{
Time evolution of dipole tilt angle in six retried runs. 
The result in the original run is plotted by gray lines. 
Re-calculations in Period 1 and 2 are shown in the upper left and right panels, respectively. 
Re-calculations in Period 3 is shown in the lower panels. 
% Range of time averaging is shown by arrows on the each panels.
Double headed arrows show periods during which data analysis is carried out.
}
\label{fig:dipole_tilt_retries}
\end{figure}
%
%
\begin{figure}[ht]
\begin{center}
\[
\begin{array}{c}
\includegraphics*[width=120mm]{Figures/temp_pvr_vrms_matsui_run_2.png}
\end{array}
\]
\end{center}
\caption{
{\color{red}
% Change of symmetry of temperature field with respect to the equator during a reversal. 
Temporal variations of symmetry with respect to the equatorial plane during polarity reversals.
% Evolution of the dipole tile is plotted in (a), evolution of symmetric and anti-symmetric components of temperature is shown in (b). 
(a) Time evolution of the dipole tile angle, (b) time evolution of equatorially symmetric and antisymmetric components of temperature, (c) volume rendering images of temperature and (d) filtered radial magnetic field on the outer boundary at $t = 13.0$, $13.5$ and $14.0$ in the unit of the magnetic diffusion time from left to right.
% Volume rendering images of temperature and filtered radial magnetic field at the outer boundary at $t = 13.0$, 13.5, and 14.0 times of the magnetic diffusion time are shown from right to left in (c) and (d), respectively.
}
}
\label{fig:temperature_rendering}
\end{figure}
%

%
\begin{figure}[ht]
\begin{center}
\[
\begin{array}{c}
\includegraphics*[width=120mm]{Figures/dipole_angle_categorized.pdf} \\
\includegraphics*[width=80mm]{Figures/g10_histgram_run1.pdf}
\end{array}
\]
\end{center}
\caption{
Time evolution of dipole tilt angle (top panel) and histogram of the square of the Gauss coefficient $|g_{1}^{0}|^2$ 
{\color{red}
with the bin size of $1.0 \times 10^{-5}$ 
}
(bottom panel). 
{\color{red}
% The stable dipole period is plotted by red dots in the top panel. 
The red and green lines are drawn during stable and reversal periods, respectively, in the top panel.
% The reversal period is shown by the green plots in the top panel and by the green shaded area in the bottom panel.
The green shaded area corresponds to the occurrence during polarity reversal periods in the bottom panel.
}
}
\label{Fig:Reversal_period_def}
\end{figure}
%

%
\begin{figure}[ht]
\begin{center}
\[
\begin{array}{c}
\includegraphics*[width=75mm]{Figures/Kpol_spectr_m.pdf} \\
\includegraphics*[width=75mm]{Figures/Ktor_spectr_m.pdf} \\
\includegraphics*[width=75mm]{Figures/Temp_spectr_m.pdf}
\end{array}
\]
\end{center}
\caption{
{\color{red}
% Time average of poloidal (top panel) and toroidal (middle panel) component of kinetic energy and mean square of the temperature spectra as a function of spherical harmonics order $m$. 
Spectra of poloidal kinetic energy (top panel), toroidal kinetic energy (middle panel), and square of temperature (bottom panel) as a funciton of spherical harmonics order, $m$. 
}
The sphere averaged component $T_{0}^{0}$ is excluded in the temperature plot.
{\color{red}
% The symmetric and anti-symmetric components with respect to the equator are plotted by the filled and open symbols, respectively. 
Spectra of equatorially symmetric and antisymmetric components are plotted by filled and open symbols, respectively.
% The results in the stable dipole and reversal periods are shown by the red and blue colors, respectively.
Spectra in stable and reversal periods are shown by red and blue colors, respectively.
}
}
\label{fig:KE_temp_spectra_m}
\end{figure}
%

%
\begin{figure}[ht]
\begin{center}
\[
\begin{array}{c}
\includegraphics*[width=120mm]{Figures/rev11_Energy_flux_evolution.png}
\end{array}
\]
\end{center}
\caption{
{\color{red}
% Evolution of energy flows during a reversal in Retry 1-1. 
% Evolution of the dipole tilt is plotted in (a), and evolution of the symmetric (red line) and anti-symmetric (black line) components of the kinetic energy are plotted in (b). 
% Energy flows for the symmetric and anti-symmetric components of kinetic energy are plotted (c) and (e), respectively. 
% Differences of the energy flows from time average for symmetric and anti-symmetric components are plotted in (d) and (f), respectively. 
% Positive energy flows indicates energy input. 
% Based on total energy fluxes in (c) and (e), energy flows with positive are plotted by solid lines, and that with negative values are plotted by dashed lines.
Time evolution of (a) the dipole tilt angle, (b) kinetic energy of the equatorially symmetric (red line) and antisymmetric (black line) components, (c) and (d) energy flows and their deviations from their time means for the equatorially symmetric component, respectively, and (e) and (f) energy flows and their deviations from their time means for the equatorially antisymmetric component, respectively.
In (c)--(f), the buoyancy flux, inertial, work of Lorentz frce, and viscous dissipation are plotted by red, green, blue, and black lines, respectively.
Positive energy flows (energy input) and negative ones (energy output) are plotted by solid and dashed lines, respectively.
}
}
\label{fig:energy_flux_evolution_retry1_1}
\end{figure}
%
%
%
\begin{figure}[ht]
\begin{center}
\[
\begin{array}{c}
\includegraphics*[width=120mm]{Figures/Averaged_flux_perturbations.pdf}
\end{array}
\]
\end{center}
\caption{
 Time and volume average of difference of energy fluxes in reversal period from those in the stable period for Period 4 to 8 and re-calculations for the Period 1 to 3. 
 $(\bvec{u}^s.\bvec{F}_{L})$ and $(\bvec{u}^a.\bvec{F}_{L})$ indicate energy fluxes into equatorilly symmetric and antisymmetric components by Lorentz force $(Pm E)^{-1} \bvec{u}^s \cdot (\bvec{J} \times \bvec{B})$ and $(Pm E)^{-1} \bvec{u}^a \cdot (\bvec{J} \times \bvec{B})$, respectively. 
 $(\bvec{u}^a.\bvec{F}_{I})$ is the energy flux to equatorially antisymmetric components by advection term $-\bvec{u}^a \cdot(\bvec{\omega} \times \bvec{u})$. $(\bvec{u}^s.\bvec{F}_{B})$ and $(\bvec{u}^a.\bvec{F}_{B})$ indicate buoyancy flux for the equatorially symmetric and antisymmetric components $Ra E^{-1} \bvec{u}^s \cdot \bvec{r} T$ and $Ra E^{-1} \bvec{u}^a \cdot \bvec{r} T$, respectively.
}
\label{Fig:Change_flux_summary_6grp}
\end{figure}
%


\begin{figure}[ht]
\begin{center}
\[
\begin{array}{cc}
\includegraphics*[width=60mm]{Figures/stable.pdf}
\includegraphics*[width=60mm]{Figures/reversal.pdf}
%\includegraphics*[width=42mm]{Figures/except.pdf}
\end{array}
\]
\end{center}
\caption{
Schematic diagram of energy flow for the equatorially symmetric and antisymmetric components of kinetic energy. 
The energy flow in the stable dipole phase is shown in the left, the change of the energy flow in the reversal phase is shown in the right. 
%And, the change of the energy flow in the exceptional case is shown in the right.
}
\label{Fig:schematic_reversal}
\end{figure}
%




\section{Discussion}

%
\begin{figure}[ht]
\begin{center}
\[
\begin{array}{ccc}
\includegraphics*[width=42mm]{Figures/stable.pdf}
\includegraphics*[width=42mm]{Figures/reversal.pdf}
\includegraphics*[width=42mm]{Figures/except.pdf}
\end{array}
\]
\end{center}
\caption{
Schematic diagram of energy flow for the symmetric and anti-symmetric components of kinetic energy. The energy flow in the stable dipole phase is shown in the left, the change of the energy flow in the reversal phase is shown in the middle. And, the change of the energy flow in the exceptional case is shown in the right.
}
\label{Fig:schematic_reversal}
\end{figure}


\section{Conclusions}
\label{section:conclusion}

We performed a dynamo simulation in a rotating spherical shell to investigate processes of the reversal of the dipole component of the magnetic field. In the present study, we investigate the equatorial symmetry of the energy fluxes for the convection during the dipole reversal from the simulation results. We perform the simulation to approximately 90 times of the magnetic diffusion time and obtained 12 reversals and 4 excursions.

First, we investigate the characteristics of the temperature field during one reversal. The results show that the equatorially anti-symmetric temperature component becomes larger than the symmetric equatorially component, and that the intense upward flow is generated in the southern hemisphere in the tangent cylinder. Looking at the power spectrum of the kinetic energy and temperature as a function of spherical harmonics order $m$ during the reversal, the equatorially anti-symmetric and axisymmetric component of the toroidal kinetic energy and temperature increase significantly during the reversal. These results suggests the todoidal anti-symmetric zonal flow is generated with sustaining the therma wind balance around the hot upwelling flow around the tangent cylinder.

We investigate the equatorial symmetry of the energy fluxes for the convection during the dipole reversal from the simulation results. The energy fluxes changes the following during the reversal: i) The energy transfer from the equatorially symmetric kinetic energy by the Lorentz force decreases, ii) energy transfer from symmetric to anti-symmetric kinetic energies are increase, and then buoyancy flux to the anti-symmetric component of the kinetic energy increases. Looking at the amplitude of the change of the energy flux to/from anti-symmetric kinetic energy, the work of the inertia term was the largest, and the second largest was the buoyancy flux. These results are the same for the 10 of 11 periods including the reversal and excursion. Considering the change of the kinetic energy during the dipole reversal, The change of the work of the inertia increase the equatorially anti-symmetric and axissymmetric component of the flow around the tangent cylinder. These results suggested that intense zonal flow is only generated in the either hemispher, and generated reversed magnetic field from th
e exsited dipolar magnetic field.

\noindent
{\bf References}
%
\begin{list}
{}{
\setlength{\parsep}{0pt}
\setlength{\itemsep}{0pt}
\setlength{\leftmargin}{1.0em}
\setlength{\itemindent}{-\leftmargin}
}
% \bibitem[\protect\citename{Busse, }1970] {Busse:1970}
\item
Busse, F.H., 1970. Thermal instabilities in rapidly rotating systems. J. Fluid Mech. 44(3), 441--460.
% \bibitem[\protect\citename{Coe and Glatzmaier, }2006] {Coe:2006}
\item
Coe, R.S., Glatzmaier, G.A., 2006. Symmetry and stability of the geomagnetic field. Geophys.\ Res.\ Lett. 33(21), doi:10.1029/2006GL027903.
%
%\bibitem[\protect\citename{Glatzmaier et al., }1999] {Glatzmaier:1999}
\item
Glatzmaier, G.A., Coe, R.S., Hongre, L., Roberts, P.H., 1999. The role of the Earth's mantle in controlling the frequency of geomagnetic reversals. Nature 401(6756), 885--890.
%
%\bibitem[\protect\citename{Glatzmaier and Roberts, }1995] {GR:1995}
\item
Glatzmaier, G.A., Roberts, P.H., 1995. A three-dimensional self-consistent computer simulation of a geomagnetic field reversal. Nature 377(6546), 203--209.
%
%\bibitem[\protect\citename{Kageyama and Sato, }1997] {Kageyama:1997}
\item
Kageyama, A., Sato, T., 1997. Generation mechanism of a dipole field by a magnetohydrodynamic dynamo. Phys.\ Rev.\ E 55(4), 4617--4626.
%
%\bibitem[\protect\citename{Kageyama et al., }1995] {Kageyama:1995}
\item
Kageyama, A., Sato, T., the Complexity Simulation Group, 1995. Computer simulation of a magnetohydrodynamic dynamo, II. Phys.\ Plasmas 2(5), 1421--1431.
%
%\bibitem[\protect\citename{Li et al., }2002] {Li:2002}
\item
Li, J., Sato, T., Kageyama, A., 2002. Repeated and sudden reversals of the dipole field generated by a spherical dynamo action. Science 295(5561), 1887--1890.
%
%\bibitem[\protect\citename{Liu and Olson, }2009] {Liu:2009}
\item
Liu, L., Olson, P., 2009. Geomagnetic dipole moment collapse by convective mixing in the core. Geophys.\ Res.\ Lett. 36(10), doi:10.1029/2009GL038130.
%
%\bibitem[\protect\citename{Matsui et al., }2014] {Matsui:2014}
\item
Matsui, H., King, E., Buffett, B., 2014. Multiscale convection in a geodynamo simulation with uniform heat flux along the outer boundary. Geochem.\ Geophys.\ Geosys. 15(8), 3212--3225, doi:10.1002/2014GC005432.
%
%\bibitem[\protect\citename{McFadden et al., }1991] {McFadden:1991}
\item
McFadden, P.L., Merrill, R.T., McElhinny, M.W., Lee, S., 1991. Reversals of the Earth's magnetic field and temporal variations of the dynamo families. J. Geophys. Res. 96(B3), 3923--3933.
%
%\bibitem[\protect\citename{Nishikawa and Kusano, }2008] {Nishikawa:2008}
\item
Nishikawa, N., Kusano, K., 2008. Simulation study of the symmetry-breaking instability and the dipole field reversal in a rotating spherical shell dynamo. Phys.\ Plasmas 15(8), 082903.
%
%\bibitem[\protect\citename{Ogg et al., } 2005] {Ogg:2005}
\item
Ogg, J.G., Agterberg, F.P., Gradstein, F.M., 2005. The Cretaceous period. In Gradstein, F.M., Ogg, J.G., Smith, A.G., editors, A Geologic Time Scale 2004, pp.\ 344--383, Cambridge University Press, Cambridge.
%
%\bibitem[\protect\citename{Olson et al., }2011] {Olson:2011}
\item
\sloppy
Olson, L.P., Glatzmaier, G.A., Coe, R.S., 2011. Complex polarity reversals in a geodynamo model. Earth Planet.\ Sci.\ Lett.\ 304(1)--(2), 168--179, doi:10.1016/j.epsl.2011.01.031.
%
%\bibitem[\protect\citename{Sarson and Jones, }1999] {Sarson:1999}
\item
Sarson, G., Jones, C., 1999. A convection driven geodynamo reversal model. Phys. Earth Planet.\ Inter. 111, 3--20.
%
%\bibitem[\protect\citename{Sreenivasan et al., }2014] {Sreenivasan:2014}
\item
Sreenivasan, B., Sahoo, S., Dhama, G., 2014. The role of buoyancy in polarity reversals of the geodynamo. Geophys.\ J. Int.\ 199(3), 1698--1709.
%
%\bibitem[\protect\citename{Takahashi et al., }2007] {TMH:2007}
\item
Takahashi, F., Matsushima, M., Honkura, Y., 2007. A numerical study on mangnetic polarity transition in an MHD dynamo model. Earth Planets Space 59(7), 665--673.
%
%\bibitem[\protect\citename{Tarduno et al., }2020] {Tarduno:2020}
\item
Tarduno, J.A., Cottrell, R.D., Bono, R.K., Oda, H., Davis, W.J., Fayek, M., van't Erve, O., Nimmo, F., Huang, W., Thern, E.R., Fearn, S., Mitra, G., Smirnov, A.V., Blackman, E.G., 2020. Paleomagnetism indicates that primary magnetite in zircon records a strong Hadean geodynamo. Proc.\ Nat.\ Acad.\ Sci. 117(5), 2309.
%
%\bibitem[\protect\citename{Wicht and Olson, }2004] {Wicht:2004}
\item
Wicht, J. and Olson, P., 2004. A detailed study of the polarity reversal mechanism in a numerical dynamo model: reversal mechanism. Geochem.\ Geophys.\ Geosys. 5(3), doi:10.1029/2003GC000602.
%
%\bibitem[\protect\citename{Winch et al., }2005] {Winch:2005}
\item
Winch, D.E., Ivers, D.J., Turner, J.P.R. and Stening, R. J., 2005. Geomagnetism and Schmidt quasi-normalization. Geophys. J. Int. 160(2), 487--504.
%
\end{list}
%\begin{thebibliography}{}
%
\bibitem[\protect\citename{Coe and Glatzmaier }2006] {Coe:2006}
Coe, R. S. \& Glatzmaier, G. A., 2006. Symmetry and stability of the geomagnetic field, {\it Geophys.\  Res.\ Lett.}, {\bf 33}(21),  doi:10.1029/2006GL027903.
%
\bibitem[\protect\citename{Glatzmaier {\it et al.} }1999] {Glatzmaier:1999}
Glatzmaier, G. A., Coe, R. S., Hongre, L., \&, Roberts, P. H., 1999, The role of the Earth's mantle in controlling the frequency of geomagnetic reversals, {\it Nature}, {\bf 401}(6756),  885--890.
%
\bibitem[\protect\citename{Glatzmaier \& Roberts }1995] {GR:1995}
Glatzmaier, G. A. \& Roberts, P. H., 1995. A three-dimensional self-consistent computer simulation of a geomagnetic field reversal, {\it Nature}, {\bf 377}(6546), 203-–209.
%
\bibitem[\protect\citename{Kageyama \& Sato }1997] {Kageyama:1997}
Kageyama, A. \& Sao, T., 1997. Generation mechanism of a dipole field by a magnetohydrodynamic dynamo, {\it Phys.\ Rev.\ E}, {\bf 55}(4), 4617--4626.
%
\bibitem[\protect\citename{Kageyama {\it et al.}, }1995] {Kageyama:1995}
Kageyama, A., Sato, T. \& the Complexity Simulation Group, 1995. Computer simulation of a magnetohydrodynamic dynamo, II, {\it Phys.\ Plasmas}, {\it 2}(5), 1421--1431.
%
\bibitem[\protect\citename{Li {\it et al.} }2002] {Li:2002}
Li, J., Sato, T. \& Kageyama, A., 2002. Repeated and sudden reversals of the dipole field generated by a spherical dynamo action, {\it Science}, {\bf 295}(5561), 1887--1890.
%
\bibitem[\protect\citename{Liu \& Olson }2009] {Liu:2009}
Liu, L. \& Olson, P., 2009. Geomagnetic dipole moment collapse by convective mixing in the core, {\it Geophys.\ Res.\ Lett.}, {\bf 36}(10), doi:10.1029/2009GL038130.
%
\bibitem[\protect\citename{Matsui {\it et al.} }2014] {Matsui:2014}
Matsui, H., King, E. \& Buffett, B., 2014. Multiscale convection in a geodynamo simulation with uniform heat flux along the outer boundary, {\it Geochem.\ Geophys.\ Geosys.}, {\bf 15}(8), 3212--3225, doi:10.1002/2014GC005432.
%
\bibitem[\protect\citename{McFadden {\it et al.} }1991] {McFadden:1991}
McFadden, P. L., Merrill, R. T., McElhinny, M. W. \& Lee, S., 1991. Reversals of the Earth's magnetic field and temporal variations of the dynamo families, \jgr {\bf 96}(B3), 3923--3933.
%
\bibitem[\protect\citename{Nishikawa \& Kusano }2008] {Nishikawa:2008}
Nishikawa, N. \& Kusano, K., 2008. Simulation study of the symmetry-breaking instability and the dipole field reversal in a rotating spherical shell dynamo, {\it Phys.\ Plasmas} {\bf 15}(8), 082903.
%
\bibitem[\protect\citename{Ogg {\it et al.} } 2005] {Ogg:2005}
Ogg, J. G., Agterberg, F. P. \& Gradstein, F. M., 2005. The Cretaceous period. In Gradstein, F. M., Ogg, J. G. \& Smith, A. G., editors, {\it A Geologic Time Scale 2004}, pp.\ 344--383, Cambridge University Press, Cambridge.
%
\bibitem[\protect\citename{Olson {\it et al.} }2011] {Olson:2011}
Olson, L. P., Glatzmaier, G. A. \& Coe, R. S., 2011. Complex polarity reversals in a geodynamo moidel, {\it Earth planet.\ Sci.\ Lett.}, {\bf 304}(1)--(2), 168--179, doi:10.1016/j.epsl.2011.01.031.
%
\bibitem[\protect\citename{Sarson \& Jones }1999] {Sarson:1999}
Sarson, G. \& Jones, C., 1999. A convection driven geodynamo reversal model, {\it Phys. Earth planet.\ Inter.}, {\bf 111}, 3--20.
%
\bibitem[\protect\citename{Sreenivasan {\it et al.}  }2014] {Sreenivasan:2014}
Sreenivasan, B., Sahoo, S. \& Dhama, G., 2014. The role of buoyancy in polarity reversals of the geodynamo, \gji {\bf 199}(3), 1698--1709.
%
\bibitem[\protect\citename{Takahashi {\it et al.}  }2007] {TMH:2007}
Takahashi, F., Matsushima, M. \& Honkura, Y., 2007. A numerical study on mangnetic polarity transition in an MHD dynamo model, \eps {\bf 59}(7), 665--673.
%
\bibitem[\protect\citename{Tarduno {\it et al.}  }2020] {Tarduno:2020}
Tarduno, J. A., Cottrell, R. D., Bono, R. K., Oda, H., Davis, W. J., Fayek, M., van't Erve, O., Nimmo, F., Huang, W., Thern, E. R., Fearn, S., Mitra, G., Smirnov, A. V. \& Blackman, E. G., 2020. Paleomagnetism indicates that primary magnetite in zircon records a strong Hadean geodynamo, {\it Proc.\ Nat.\ Acad.\ Sci.}, {\bf 117}(5), 2309.
%
\bibitem[\protect\citename{Wicht \& Olson }2004] {WIcht:2004}
Wicht, J. \& Olson, P., 2004. A detailed study of the polarity reversal mechanism in a numerical dynamo model: reversal mechanism, {\it Geochem.\ Geophys.\ Geosys.}, {\bf 5}(3), doi:10.1029/2003GC000602.
%
\bibitem[\protect\citename{Winch {\it et al.}  }2005] {Winch:2005}
Winch, D. E., Ivers, D. J., Turner, J. P. R. \& Stening, R. J., 2005. Geomagnetism and Schmidt quasi-normalization, \gji {\bf 160}(2), 487--504.
%
\end{thebibliography}

\end{document}
