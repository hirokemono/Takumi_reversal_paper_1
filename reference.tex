\begin{thebibliography}{}
%
\bibitem[\protect\citename{Busse, }1970] {Busse:1970}
Busse, F.H., 1970. Thermal instabilities in rapidly rotating systems. {\it J. Fluid Mech.}, 44(3), 441--460.
%
\bibitem[\protect\citename{Cande and Kent, }1995] {Cande:1995}
Cande, S.C. and Kent, D.V., 1995. Revised calibration of the geomagnetic polarity timescale for the Late Cretaceous and Cenozoic, {\it J. of Geophys. Res.}, {\bf 100}, B4, 6093--6095, https://doi.org/10.1029/94JB03098.
%
\bibitem[\protect\citename{Chiristensen and Aubert, }2006]{aubert:2006}
Christensen, U.R. and Aubert, J., 2006. Scaling properties of convection-driven dynamos in rotating spherical shells and application to planetary fields. {\it Geophys. J. Int.}, {\bf 166}, 97–114.
%   
\bibitem[\protect\citename{Coe and Glatzmaier, }2006] {Coe:2006}
Coe, R.S., Glatzmaier, G.A., 2006. Symmetry and stability of the geomagnetic field. {\it Geophys.\ Res.\ Lett.}, 33(21), doi:10.1029/2006GL027903.
%
\bibitem[\protect\citename{Driscoll and Olson, }2009] {driscoll:2009}
Driscoll, P. and Olson, P., Effects of buoyancy and rotation on the polarity reversal frequency of gravitationally driven numerical dynamos. {\it Geophys. J. Int.}, doi:10.111/j.1365-246X.2009.04234.x.
%
\bibitem[\protect\citename{Glatzmaier et al., }1999] {Glatzmaier:1999}
Glatzmaier, G.A., Coe, R.S., Hongre, L., Roberts, P.H., 1999. The role of the Earth's mantle in controlling the frequency of geomagnetic reversals. {\it Nature}, 401(6756), 885--890.
%
\bibitem[\protect\citename{Glatzmaier and Roberts, }1995] {GR:1995}
Glatzmaier, G.A., Roberts, P.H., 1995. A three-dimensional self-consistent computer simulation of a geomagnetic field reversal. {\it Nature}, 377(6546), 203--209.
%
\bibitem[\protect\citename{Kageyama and Sato, }1997] {Kageyama:97c}
Kageyama, A., Sato, T., 1997. Generation mechanism of a dipole field by a magnetohydrodynamic dynamo. {\it Phys.\ Rev.\ E}, 55(4), 4617--4626.
%
\bibitem[\protect\citename{Kageyama et al., }1995] {Kageyama:1995}
Kageyama, A., Sato, T., the Complexity Simulation Group, 1995. Computer simulation of a magnetohydrodynamic dynamo, II. {\it Phys.\ Plasmas}, 2(5), 1421--1431.
%
\bibitem[\protect\citename{Li et al., }2002] {Li:2002}
Li, J., Sato, T., Kageyama, A., 2002. Repeated and sudden reversals of the dipole field generated by a spherical dynamo action. {\it Science}, 295(5561), 1887--1890.
%
%\bibitem[\protect\citename{Liu and Olson, }2009] {Liu:2009}
%Liu, L., Olson, P., 2009. Geomagnetic dipole moment collapse by convective mixing in the core. {\it Geophys.\ Res.\ Lett.}, 36(10), doi:10.1029/2009GL038130.
%
\bibitem[\protect\citename{Matsui et al., }2014] {Matsui:2014}
Matsui, H., King, E., Buffett, B., 2014. Multiscale convection in a geodynamo simulation with uniform heat flux along the outer boundary. {\it Geochem.\ Geophys.\ Geosys.}, 15(8), 3212--3225, doi:10.1002/2014GC005432.
%
%\bibitem[\protect\citename{McFadden et al., }1991] {McFadden:1991}
%McFadden, P.L., Merrill, R.T., McElhinny, M.W., Lee, S., 1991. Reversals of the Earth's magnetic field and temporal variations of the dynamo families. {\it J. Geophys. Res.}, 96(B3), 3923--3933.
%
\bibitem[\protect\citename{Nishikawa and Kusano, }2008] {Nishikawa:2008}
Nishikawa, N., Kusano, K., 2008. Simulation study of the symmetry-breaking instability and the dipole field reversal in a rotating spherical shell dynamo. {\it Phys.\ Plasmas}, 15(8), 082903.
%
%\bibitem[\protect\citename{Ogg et al., } 2005] {Ogg:2005}
%Ogg, J.G., Agterberg, F.P., Gradstein, F.M., 2005. The Cretaceous period. In Gradstein, F.M., Ogg, J.G., Smith, A.G., editors, A Geologic Time Scale 2004, pp.\ 344--383, Cambridge University Press, Cambridge.
%
\bibitem[\protect\citename{Olson and Christensen, }2006] {Olson:2006}
Olson, P., and Christensen, U.R., 2011. Dipole moment scaling for convection-driven planetary dynamos. {\it Earth Planet.\ Sci.\ Lett.}, 250, 561--571.
%
\bibitem[\protect\citename{Olson et al., }2011] {Olson:2011}
Olson, L.P., Glatzmaier, G.A., Coe, R.S., 2011. Complex polarity reversals in a geodynamo model. Earth Planet.\ Sci.\ Lett.\ 304(1)--(2), 168--179, doi:10.1016/j.epsl.2011.01.031.
%
\bibitem[\protect\citename{Sarson and Jones, }1999] {Sarson:1999}
Sarson, G., Jones, C., 1999. A convection driven geodynamo reversal model. Phys.\ Earth Planet.\ Inter.\ 111, 3--20.
%
\bibitem[\protect\citename{Sreenivasan et al., }2014] {Sreenivasan:2014}
Sreenivasan, B., Sahoo, S., Dhama, G., 2014. The role of buoyancy in polarity reversals of the geodynamo. Geophys.\ J. Int.\ 199(3), 1698--1709.
%
\bibitem[\protect\citename{Takahashi et al., }2007] {TMH:2007}
Takahashi, F., Matsushima, M., Honkura, Y., 2007. A numerical study on mangnetic polarity transition in an MHD dynamo model. Earth Planets Space 59(7), 665--673.
%
\bibitem[\protect\citename{Tarduno et al., }2020] {Tarduno:2020}
Tarduno, J.A., Cottrell, R.D., Bono, R.K., Oda, H., Davis, W.J., Fayek, M., van't Erve, O., Nimmo, F., Huang, W., Thern, E.R., Fearn, S., Mitra, G., Smirnov, A.V., Blackman, E.G., 2020. Paleomagnetism indicates that primary magnetite in zircon records a strong Hadean geodynamo. Proc.\ Nat.\ Acad.\ Sci.\ 117(5), 2309.
%
\bibitem[\protect\citename{Wicht and Olson, }2004] {Wicht:2004}
Wicht, J. and Olson, P., 2004. A detailed study of the polarity reversal mechanism in a numerical dynamo model: reversal mechanism. Geochem.\ Geophys.\ Geosys.\ 5(3), doi:10.1029/2003GC000602.
%
\bibitem[\protect\citename{Winch et al., }2005] {Winch:2005}
Winch, D.E., Ivers, D.J., Turner, J.P.R. and Stening, R. J., 2005. Geomagnetism and Schmidt quasi-normalization. Geophys.\ J. Int.\ 160(2), 487--504.

\end{thebibliography}