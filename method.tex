\section{Method}
\label{section:method}

\subsection{Numerical method}

We perform numerical simulations of an MHD dynamo to investigate mechanism related to polarity reversals of the Earth's magnetic field.
The fluid outer core, in which dynamo action occurs, is represented by a spherical shell rapidly rotating with angular velocity $\bvec{\Omega} = \Omega \hat{\bvec{z}}$, where $\hat{\bvec{z}}$ is the unit vector aligned to the rotation axis.
The spherical shell with inner and outer radii, $r_i$ and $r_o$, respectively, and $r_i / r_o = 0.35$, is filled with an electrically conducting Boussinesq fluid, which is assumed to have constant kinetic viscosity $\nu$, thermal diffusivity $\kappa$, and magnetic diffusivity $\eta$.
This leads to an equation of continuity as
%
\begin{equation}
\nabla \cdot \bvec{u} = 0,
\label{eq:divu=0}
\end{equation}
%
where $\bvec{u}$ is the velocity field of core fluid.
The magnetic field, $\bvec{B}$, satisfies the Gauss's law given as
%
\begin{equation}
\nabla \cdot \bvec{B} = 0.
\label{eq:divB=0}
\end{equation}
%
The other nondimensional governing equations for the present MHD dynamo driven by thermally convective motions are derived as
%
\begin{equation}
\begin{array}{l}
\displaystyle
E \left( \frac{\partial\bvec{u}}{\partial t} +
 \bvec{\omega} \times \bvec{u} \right) =
 - \nabla \left( P + E\frac{1}{2} \bvec{u}^2 \right)
 + E \nabla^2 \bvec{u}
% \nonumber 
\\
\displaystyle
\hspace*{\fill}
 -2 \hat{\bvec{z}} \times \bvec{u}
 + Ra_f T \frac{\bvec{r}}{r_o}
 + \frac{1}{Pm} \bvec{J} \times \bvec{B} ,
\end{array}
\label{eq:momentum}
\end{equation}
%
\begin{equation}
\frac{\partial \bvec{B}}{\partial t} =
 \nabla \times (\bvec{u} \times \bvec{B} )
 + \frac{1}{Pm} \nabla^2 \bvec{B} ,
\label{eq:induction}
\end{equation}
%
\begin{equation}
\frac{\partial T}{\partial t} 
 + ( \bvec{u} \cdot \nabla ) T =
 \frac{1}{Pr} \nabla^2 T ,
\label{eq:heat}
\end{equation}
%
where $t$ is the time, $\bvec{\omega}$ is the vorticity, $P$ is the pressure, $\bvec{r}$ is the position vector, and $\bvec{J}$ is the electric current density.
The length, time, pressure, temperature, and magnetic field are respectively scaled by $D$, $D^2/\nu$, $\nu^2 /D^2$, $\beta_o D$, and $(\rho \mu_0 \Omega \eta )^{1/2}$,
where $D = r_o - r_i$, $\rho$, $\mu_0$, and $\beta_o$ are the thickness of the outer core, the mean density of the 
% {\color{teal}
% electrically
% } 
conductive fluid, the magnetic permeability of vacuum, and the temperature gradient at the core-mantle boundary (CMB), respectively.
The dimensionless numbers included in the governing equations are the Rayleigh number, $Ra_f$, the Ekman number, $E$, the Prandtl number, $Pr$, and the magnetic Prandtl number, $Pm$, which are defined as
%
\begin{equation}
% Ra_f = \frac{\alpha g_o \beta_o r_{o}^{2}}{\nu \Omega},~~
% {\color{teal}
Ra_f = \frac{\alpha g_o \beta_o D^{2}}{\nu \Omega},~~
E = \frac{\nu}{\Omega D^2},~~
Pr = \frac{\nu}{\kappa},~~ \mbox{and \ }
Pm = \frac{\nu}{\eta},
\label{eq:dimensionless_numbers}
\end{equation}
%
where $\alpha$ is the coefficient of thermal volume expansion, and $g_o$ is the amplitude of gravity at $r = r_o$.

The solenoidal vector fields, $\bvec{u}$ and $\bvec{B}$, are decomposed into the toroidal and poloidal components as
%
\begin{equation}
    \bvec{u} (\bvec{r}, t) = \nabla \times ( u_T (\bvec{r}, t) \hat{\bvec{r}} ) + \nabla \times \nabla \times ( u_S (\bvec{r}, t) \hat{\bvec{r}} ),
\label{eq:uT_uS}
\end{equation}
%
\begin{equation}
    \bvec{B} (\bvec{r}, t) = \nabla \times ( B_T (\bvec{r}, t) \hat{\bvec{r}} ) + \nabla \times \nabla \times ( B_S (\bvec{r}, t) \hat{\bvec{r}} ),
\label{eq:BT_BS}
\end{equation}
%
where $\hat{\bvec{r}}$ is the radial unit vector.
Toroidal and poloidal scalar functions for the velocity and magnetic fields, $u_T (\bvec{r}, t)$, $u_S (\bvec{r}, t)$, $B_T (\bvec{r}, t)$, and $B_S (\bvec{r}, t)$, 
are expanded into spherical harmonics in the horizontal directions. For example, $u_{S} (\bvec{r}, t)$ is expanded as
%
\begin{equation}
    u_{S} (\bvec{r}, t) = \sum_{l=1}^{L_{\rm max}} \sum_{m=-l}^{l} u_{Sl}^{\ m} (r, t) Y_l^{|m|} (\theta, \phi),
\label{eq:Us_expansion}
\end{equation}
%
where $L_{\rm max}$ is the truncation of spherical harmonic expansion, and
%
\begin{equation}
Y_l^{|m|} (\theta, \phi) = \left\{
 \begin{array}{ll}
 P_l^m(\cos\theta)\cos m\phi & (m = 0, 1, 2, \cdots, l)
 \\
 P_l^{|m|}(\cos\theta)\sin |m|\phi & (m = -1, -2, \cdots, -l) ,
 \end{array}
\right.
\label{eq:def_of_Ylm}
\end{equation}
%
where, $P_l^m (\cos \theta )$ is a Schmidt quasi-normalized associated Legendre polynomial with degree $l$ and order $m$ (Winch et al., 2005). 
% \cite{Winch:2005}).
The temperature field is also expanded into the spherical harmonics coefficients as 
%
\begin{equation}
    T(\bvec{r}, t) = \sum_{l=0}^{L_{\rm max}} \sum_{m=-l}^{l} T_l^m (r, t) Y_l^{|m|} (\theta, \phi).
\label{eq:T_expansion}
\end{equation}
%

In the radial direction, scalar functions, $u_T (\bvec{r}, t)$, $u_S (\bvec{r}, t)$, $B_T (\bvec{r}, t)$, $B_S (\bvec{r}, t)$, and $T(\bvec{r}, t)$, are differentiated by the second-order finite difference method. 
To obtain high spatial resolution near the boundaries and a smooth transition of grid spacing, the so-called Chebyshev grid is used for the radial grid points defined as
%
\begin{equation}
r_n = r_i + \frac{r_o - r_i}{2} \left\{ 1 - \cos \left( \pi \frac{n-1}{N_r-1} \right) \right\} ~~\;\;\;\; (n = 1, \cdots , N_r) ,
\label{eq:def_of_rn}
\end{equation}
%
where $N_r$ is the number of radial grids.

For the time integration, the Crank-Nicolson scheme is adopted for the diffusion terms, and the second order Adams-Bashforth scheme is used for the other terms.

The no-slip condition for the velocity field is imposed at impermeable boundary surfaces, and the inner core is assumed to co-rotate with the mantle, which leads to $\bvec{u} = {\bf 0}$.
The regions outside the spherical shell corresponding to the inner core and the mantle are assumed to be electrical insulators for simplicity, so that the magnetic field in the spherical shell is continuous to a potential field both at $r = r_i$ and $r = r_o$.
%An 
% {\color{red}
A
% }
uniform temperature gradient at the CMB is imposed as $-\partial T_0^0 / \partial r |_{r = r_o} = r_{o}^{-2} = 0.4225$. 
To satisfy $T_0^0 (r, t)$ with $\nabla^2 T_0^0 = 0$,
the temperature gradient at the inner core boundary (ICB) is determined from the balance between the heat flux into at $r = r_i$ and out of $r = r_o$. 
Consequently, the uniform temperature gradient at the ICB is set to $-\partial T_0^0 / \partial r |_{r = r_i} = r_i^{-2}$.

The initial condition for numerical simulations is set as follows; $\bvec{u} = {\bf 0}$ for the velocity field, a geocentric magnetic dipole moment whose tilt angle from the rotation axis is $\pi / 4$, and a component of degree 4 and order 4 of spherical harmonics for the temperature as adopted by a dynamo benchmark (Christensen et al., 2001).
%The dimensionless parameters are set as $Ra_f = 2000$, $E = 6 \times 10^{-4}$, $Pr = 1$, and $Pm = 5$ on the basis of results by Sreenivasan et al.\ (2014). % \cite{Sreenivasan:2014}.
% {\color{teal}
The dimensionless parameters are set as $Ra_f = 2000$, $E = 6.0 \times 10^{-4}$, $Pr = 1.0$, and $Pm = 5.0$ on the basis of results by Sreenivasan et al.\ (2014).
% }

\subsection{Equatorial symmetry}

We investigate possible variations in equatorial symmetry of the velocity and magnetic fields in relation to polarity reversals.
Dimensionless kinetic energy density and dimensionless magnetic energy density are respectively defined as
%
\begin{equation}
E_{\rm kin} = \frac{1}{V}
  \int_V \frac{1}{2} \bvec{u}^2 d V ,
\label{eq:Ekin}
\end{equation}
%
\begin{equation}
E_{\rm mag} = \frac{1}{V Pm E}
  \int_V \frac{1}{2} \bvec{B}^2 d V ,
\label{eq:Emag}
\end{equation}
%
where integrals are carried out over the volume of spherical shell, $V$.
Temporal variations of kinetic and magnetic energy densities can be derived from (\ref{eq:momentum}) and (\ref{eq:induction}), respectively.
Their energy equations are given as
%
\begin{eqnarray}
\frac{\partial}{\partial t}
 \int_V \frac{|\bvec{u}|^2}{2} d V
 &=& \int_V \left\{
    - |\bvec{\omega}|^2 
    + \frac{Ra_f}{E}T \bvec{u}\cdot 
    \frac{\bvec{r}}{r_o}
      \right.
\nonumber \\
 & & \left.
    - \bvec{u} \cdot (\bvec{\omega}\times\bvec{u})
    + \frac{1}{Pm E} \bvec{u} \cdot
                     (\bvec{J} \times \bvec{B})
      \right\} d V,
\label{eq:energy_u}
\end{eqnarray}
%
\begin{eqnarray}
\frac{1}{Pm E}\frac{\partial}{\partial t}
 \int_V \frac{|\bvec{B}|^2}{2} d V
  & = & 
- \frac{1}{Pm E}\int_V \left\{
      \bvec{u} \cdot ( \bvec{J} \times \bvec{B} )
\right. \nonumber \\
 & &    \left.
    + \frac{1}{Pm} |\bvec{J}|^2
    + \nabla \cdot ( \bvec{E} \times \bvec{B} ) 
      \right\} d V.
\label{eq:energy_B}
\end{eqnarray}
%
Any vector can be divided into equatorially symmetric and antisymmetric constituents.
% {\color{teal}
For example, 
% }
we represent the velocity and magnetic fields as
%
\begin{equation}
\bvec{u} = \bvec{u}^s + \bvec{u}^a, ~~~
\bvec{B} = \bvec{B}^s + \bvec{B}^a,
\label{eq:eqas}
\end{equation}
%
where superscripts $s$ and $a$ denote equatorially symmetric and antisymmetric fields, respectively.
We then derive the energy equations for the equatorially symmetric and antisymmetric velocity field, respectively, given as
%
\begin{eqnarray}
\displaystyle
\frac{\partial}{\partial t}
 \int \frac{|\bvec{u}^s|^2}{2} d V 
 &=& \int \left\{
      \frac{Ra_f}{E}T^s \bvec{u}^s\cdot \frac{\bvec{r}}{r_{o}}
     \right.
\nonumber \\
& & \displaystyle
\hspace*{2em}
     \left.
    + \frac{1}{Pm E} \bvec{u}^s \cdot
                (\bvec{J}^s \times \bvec{B}^a
                +\bvec{J}^a \times \bvec{B}^s)
      \right.
\nonumber \\
& &\displaystyle
\hspace*{3em}
      \left.
    - \bvec{u}^s \cdot 
       (\bvec{\omega}^s \times \bvec{u}^a)
    - |\bvec{\omega}^a|^2 
      \right\} d V,
\label{eq:energy_us}
\end{eqnarray}
%
\begin{eqnarray}
\displaystyle
\frac{\partial}{\partial t}
 \int \frac{|\bvec{u}^a|^2}{2} d V 
 & = & \displaystyle
% \hspace*{1em}
\int \left\{
      \frac{Ra_f}{E}T^a \bvec{u}^a\cdot \frac{\bvec{r}}{r_{o}}
     \right.
\nonumber \\
& & \displaystyle
\hspace*{2em}
     \left.
    + \frac{1}{Pm E} \bvec{u}^a \cdot
                (\bvec{J}^a \times \bvec{B}^a
                +\bvec{J}^s \times \bvec{B}^s)
      \right.
\nonumber \\
& & \displaystyle
\hspace*{3em}
      \left.
%      {\color{red} - } 
     - \bvec{u}^a \cdot 
       (\bvec{\omega}^s \times \bvec{u}^s)
    - |\bvec{\omega}^s|^2 
      \right\} d V.
\label{eq:energy_ua}
\end{eqnarray}
%
In the same way, the energy equations for the equatorially symmetric and antisymmetric magnetic field are respectively obtained as
%
\begin{eqnarray}
\displaystyle
\frac{1}{Pm E}\frac{\partial}{\partial t}
 \int \frac{|\bvec{B}^s|^2}{2} d V
 & = & - \frac{1}{Pm E}\int \left\{
      \bvec{u}^a \cdot 
          ( \bvec{J}^a \times \bvec{B}^a )
    + \bvec{u}^s \cdot 
          ( \bvec{J}^a \times \bvec{B}^s )
    \right.
\nonumber \\
\hspace{3em}
&& \displaystyle
    \left.
    + \frac{1}{Pm} |\bvec{J}^a|^2
    + \nabla \cdot ( \bvec{E}^a \times \bvec{B}^s ) 
      \right\} d V,
% \end{array}
\label{eq:energy_Bs}
\end{eqnarray}
%
and 
%
\begin{eqnarray}
\displaystyle
\frac{1}{Pm E}\frac{\partial}{\partial t}
 \int \frac{|\bvec{B}^a|^2}{2} d V 
& = & \displaystyle
- \frac{1}{Pm E}\int \left\{
      \bvec{u}^a \cdot 
          ( \bvec{J}^s \times \bvec{B}^s )
    + \bvec{u}^s \cdot ( \bvec{J}^s \times \bvec{B}^a )
    \right.
\nonumber \\
\hspace{3em}
 & & \displaystyle
    \left.
    + \frac{1}{Pm} |\bvec{J}^s|^2
    + \nabla \cdot ( \bvec{E}^s \times \bvec{B}^a ) 
      \right\} d V.
\label{eq:energy_Ba}
\end{eqnarray}

%The right-hand-sides of (\ref{eq:energy_us}) and (\ref{eq:energy_ua}) show energy transfer due to respective forces as follows.
%The first terms corresponding to the work by buoyancy mean that equatorially symmetric and antisymmetric temperature fields contribute to kinetic energy for the equatorially symmetric and antisymmetric velocity fields, respectively.
%{\color{blue}
%The second terms are the work by the Lorentz force which show energy transfer between kinetic and magnetic energies. The energy transfers by the Lorentz force from the symmetric component of the kinetic energy to the antisymmetric and symmetric components of the magnetic energy are described by $- \bvec{u}^s \cdot (\bvec{J}^s \times \bvec{B}^a)$ and $- \bvec{u}^s \cdot (\bvec{J}^a \times \bvec{B}^s)$ in (\ref{eq:energy_us}), respectively. 
%From the antisymmetric component of the kinetic energy, the energy transfers by the Lorentz force from the kinetic energy to the antisymmetric and symmetric components of the magnetic energy are described by $- \bvec{u}^a \cdot (\bvec{J}^s \times \bvec{B}^s)$ and $- \bvec{u}^a \cdot (\bvec{J}^a \times \bvec{B}^a)$ in (\ref{eq:energy_ua}), respectively. 
%Consequently, the $- \bvec{u}^s \cdot (\bvec{J}^a \times \bvec{B}^s)$ and $- \bvec{u}^a \cdot (\bvec{J}^a \times \bvec{B}^a)$ appear in (\ref{eq:energy_Bs}) as the energy input to the symmetric component of the magnetic field by the magnetic induction, and $- \bvec{u}^s \cdot (\bvec{J}^s \times \bvec{B}^a)$ and $- \bvec{u}^a \cdot (\bvec{J}^s \times \bvec{B}^s)$ appear in (\ref{eq:energy_Ba}) as the energy input to the antisymmetric component of the magnetic field by the magnetic induction.

%The work of inertia term $-\bvec{u} \cdot \left(\bvec{\omega} \times \bvec{u} \right)$ in equation (\ref{eq:energy_u}) is also expanded into that for symmetric and ant-symmetric component in equations (\ref{eq:energy_us}) and (\ref{eq:energy_ua}) same as the work of Lorentz force. However, the terms $- \bvec{u}^s \cdot (\bvec{\omega}^a \times \bvec{u}^s) = - \bvec{\omega}^a \cdot (\bvec{u}^s \times \bvec{u}^s) $ and $- \bvec{u}^a \cdot (\bvec{\omega}^a \times \bvec{u}^a) = - \bvec{\omega}^a \cdot (\bvec{u}^a \times \bvec{u}^a)$ are equal to be zero. The remaining terms $-\bvec{u}^s \cdot (\bvec{\omega}^s \times \bvec{u}^a)$ in equation (\ref{eq:energy_us}) and $- \bvec{u}^a \cdot (\bvec{\omega}^s \times \bvec{u}^s)$ in equation (\ref{eq:energy_ua}) represent the kinetic energy transfer between the symmetric and anti-symmetric components by inertia because $- \bvec{u}^a \cdot (\bvec{\omega}^s \times \bvec{u}^s)$ in equation (\ref{eq:energy_ua}) is equal to be $\bvec{u}^s \cdot (\bvec{\omega}^s \times \bvec{u}^a)$. Consequently, the inertia term does not contribute to total kinetic energy.
%}
%The second terms, the work by the Lorentz force, show energy transfer between kinetic and magnetic energies, as they are also found in (\ref{eq:energy_Bs}) and (\ref{eq:energy_Ba}); 
%that is, $\bvec{u}^s \cdot (\bvec{J}^s \times \bvec{B}^a)$ corresponds to energy transfer between $\bvec{u}^s$ and $\bvec{B}^a$, $\bvec{u}^s \cdot (\bvec{J}^a \times \bvec{B}^s)$ to that between $\bvec{u}^s$ and $\bvec{B}^s$, $\bvec{u}^a \cdot (\bvec{J}^a \times \bvec{B}^a)$ to that between $\bvec{u}^a$ and $\bvec{B}^a$, and $\bvec{u}^a \cdot (\bvec{J}^s \times \bvec{B}^s)$ to that between $\bvec{u}^a$ and $\bvec{B}^s$.
%In other words, $-\bvec{u}^s \cdot (\bvec{J}^s \times \bvec{B}^a)$ contributes to temporal variations of $\bvec{B}^a$ caused by $\bvec{u}^s$, 
%$-\bvec{u}^s \cdot (\bvec{J}^a \times \bvec{B}^s)$ to those of $\bvec{B}^s$ by $\bvec{u}^s$,
%$-\bvec{u}^a \cdot (\bvec{J}^a \times \bvec{B}^a)$ to those of $\bvec{B}^a$ by $\bvec{u}^a$, and 
%$-\bvec{u}^a \cdot (\bvec{J}^s \times \bvec{B}^s)$ to those of $\bvec{B}^s$ by $\bvec{u}^a$, as found in (\ref{eq:energy_Bs}) and (\ref{eq:energy_Ba}).
% The third terms in the right-hand-sides of (\ref{eq:energy_us}) and (\ref{eq:energy_ua}) express energy transfer between $\bvec{u}^s$ and $\bvec{u}^a$ due to the advection, which does not contribute to total kinetic energy.
% {\color{teal}
The right-hand-sides (RHS) of (\ref{eq:energy_us})-- (\ref{eq:energy_Ba}) show the energy transfer due to respective factors, and they are summarized as follows.
\begin{enumerate}
\item 
The first terms in the RHS of (\ref{eq:energy_us}) and (\ref{eq:energy_ua}) correspond to the work by buoyancy, and they mean that equatorially symmetric and antisymmetric temperature fields, $T^s$ and $T^a$, contribute to kinetic energy for the equatorially symmetric and antisymmetric velocity fields, $\bvec{u}^s$ and $\bvec{u}^a$, respectively.

\item 
The second terms in the RHS of (\ref{eq:energy_us}) and (\ref{eq:energy_ua}) correspond to the work by the Lorentz force, and they show energy transfer between kinetic and magnetic energies.
The terms, $+\bvec{u}^s \cdot (\bvec{J}^s \times \bvec{B}^a)$ and $+\bvec{u}^s \cdot (\bvec{J}^a \times \bvec{B}^s)$, show the energy transfer from the kinetic energy for the equatorially symmetric flow, $\bvec{u}^s$, to the magnetic energy for the equatorially antisymmetric and symmetric magnetic fields, $\bvec{B}^a$  and $\bvec{B}^s$, respectively. 
Consequently, $-\bvec{u}^s \cdot (\bvec{J}^s \times \bvec{B}^a)$ in (\ref{eq:energy_Ba}) and $-\bvec{u}^s \cdot (\bvec{J}^a \times \bvec{B}^s)$ in (\ref{eq:energy_Bs}) appear as the energy transfer to the magnetic energy for $\bvec{B}^a$ and $\bvec{B}^s$ by the magnetic induction, respectively.
In the same way, the terms, $+\bvec{u}^a \cdot (\bvec{J}^s \times \bvec{B}^s)$ and $+\bvec{u}^a \cdot (\bvec{J}^a \times \bvec{B}^a)$, show the energy transfer from the kinetic energy for $\bvec{u}^a$ to the magnetic energy for $\bvec{B}^s$ and $\bvec{B}^a$ by the magnetic induction, respectively.
Therefore, $-\bvec{u}^a \cdot (\bvec{J}^s \times \bvec{B}^s)$ in (\ref{eq:energy_Ba}) and $-\bvec{u}^a \cdot (\bvec{J}^a \times \bvec{B}^a)$ in (\ref{eq:energy_Bs}) appear as the energy transfer to the magnetic energy for $\bvec{B}^a$ and $\bvec{B}^s$ by the magnetic induction, respectively.

\item 
The third terms in the RHS of (\ref{eq:energy_us}) and (\ref{eq:energy_ua}) correspond to the work by inertia, and they do not contribute to total kinetic energy.
The term, $-\bvec{u} \cdot (\bvec{\omega} \times \bvec{u})$, in (\ref{eq:energy_u}) is expressed in terms of $\bvec{u}^s$ and $\bvec{u}^a$, but $-\bvec{u}^s \cdot (\bvec{\omega}^a \times \bvec{u}^s) = -\bvec{\omega}^a \cdot (\bvec{u}^s \times \bvec{u}^s)$ and $-\bvec{u}^a \cdot (\bvec{\omega}^a \times \bvec{u}^a) = -\bvec{\omega}^a \cdot (\bvec{u}^a \times \bvec{u}^a)$ vanish. 
The remaining terms, $-\bvec{u}^s \cdot \bvec{\omega}^a \times \bvec{u}^a$ in (\ref{eq:energy_us}) and $-\bvec{u}^a \cdot (\bvec{\omega}^s \times \bvec{u}^a)$ in (\ref{eq:energy_ua}), represent the energy transfer between the kinetic energy for $\bvec{u}^s$ and $\bvec{u}^a$ due to inertia as found by $-\bvec{u}^a \cdot (\bvec{\omega}^s \times \bvec{u}^s) = \bvec{u}^s \cdot (\bvec{\omega}^s \times \bvec{u}^a)$.

\item 
The fourth terms in the RHS of (\ref{eq:energy_us}) and (\ref{eq:energy_ua}) indicate the viscous dissipation.

\item 
The third terms in the RHS of (\ref{eq:energy_Bs}) and (\ref{eq:energy_Ba}) indicate the Ohmic dissipation.

\item 
The fourth terms in the RHS of (\ref{eq:energy_Bs}) and (\ref{eq:energy_Ba}) correspond to the Poynting flux which can be expressed, for example, as
\begin{equation}
-\int \nabla \cdot
  ( \bvec{E}^a \times \bvec{B}^s ) d V =
 \oint (\bvec{E}^a \times \bvec{B}^s )
   \cdot \hat{\bvec{r}} d S .
\label{eq:Poynting}
\end{equation}
\end{enumerate}
% }
%The fourth terms indicate the viscous dissipation. 
%The third terms in the right-hand-sides of (\ref{eq:energy_Bs}) and (\ref{eq:energy_Ba}) indicate the Ohmic dissipation.
%The forth terms correspond to the Poynting flux which can be expressed, for example, as
%
%\begin{equation}
%-\int \nabla \cdot
%  ( \bvec{E}^a \times \bvec{B}^s ) d V =
% \oint (\bvec{E}^a \times \bvec{B}^s )
%   \cdot \hat{\bvec{r}} d S .
%\label{eq:Poynting}
%\end{equation}
%

We implemented subroutines to evaluate these volume averaged energy transfers decomposed to the equatorial symmetry into a numerical dynamo code, Calypso Ver.~1.2 (Matsui et al., 2014) 
% \cite{Matsui:2014}) 
and performed dynamo simulations with polarity reversals.
%
The source code and documents of Calypso Ver.~1.2 can be found in the following URL;\\
https://github.com/geodynamics/calypso.\\
