\section{Discussion}
\label{section:discussion}

We restarted numerical simulations a few times from the identical snapshot using the same dimensionless numbers and the time step $\Delta t$.
However, we obtained results different from the original one; the solutions are not exactly identical
as found in the evolution of the dipole tilt angle after a few magnetic diffusion time (see Fig.~\ref{fig:dipole_tilt_retries}). 
Calypso chooses the fastest algorithms for the Legendre and Fourier transforms at the initialization process in numerical simulations. 
The different order of summation can change the last digit of data. 
As a result, numerical errors accumulate with the time integration. 
{\color{red}
% However, the time average the results are almost identical, and the dipole reversal happens in the both original and retried cases.
We found, however, that polarity reversals occur in the same way, and that the statistical results are similar to the original one.
% We consider that the present results have a limitation of a numerical modeling, but that these results are still feasible.
Hence, we consider that the present results are substantial and feasible.
}

The dimensionless numbers adopted in the present study are based on those used by Sreenivasan et al.\ (2014). % \cite{Sreenivasan:2014}. 
As shown in Fig.~\ref{Fig:more_cases}, however, we obtained a solution of multipolar dominated dynamo for $Ra_f = 2700$, although Sreenivasan et al.\ (2014) obtained Earth-like dipole reversals with the same Rayleigh number (it is noted that Sreennivasan et al.\ (2014) defined the Rayleigh number as $Ra_f = \alpha g_o \beta_o D^2 / 2\Omega \nu$).
The difference is likely to originate from the different thermal boundary condition at the inner boundary. 
We fixed the heat flux at the inner boundary, while Sreenivasan et al.\ (2014) fixed the temperature at the inner boundary. 
{\color{red}
The uniformly fixed temperature boundary condition forces the temperature to be equatorially symmetric at the inner boundary, although the temperature can be equatorially antisymmetric away from the inner boundary.
}
Consequently, Sreenivasan et al.\ (2014) required a larger Rayleigh number to give rise to the equatorially antisymmetric temperature and flow in turn. 
% 以下,よくわかりませんでした... Rewritten <- Danke
{\color{green}
However, no polarity reversal can be found in the cases of $Ra_f = 1500$ in the present study and $Ra_f = 1620$ in Sreenivasan et al.\ (2014) (See left panel of Fig.~\ref{Fig:more_cases}). 
% The result suggests that the effects of the thermal boundary condition at the ICB on the lower bound of $Ra_f$ to give rise to the polarity reversal is not significant.
The result suggests that the thermal boundary condition at the ICB does not have a significant effect on the lower bound of $Ra_f$ to give rise to the polarity reversal.
%However, we expect that the effects of the homogeneous thermal boundary condition is not significant because no reversal is obtained for $Ra_f = 1500$ in the present study and for $Ra_f = 1620$ in Sreenivasan et al.\ (2014) (See left panel of Fig.~\ref{Fig:more_cases}).
}

\begin{figure}[ht]
\begin{center}
\[
\begin{array}{c}
Ra_f = 1500  \\
\includegraphics*[width=90mm]{Figures/sph_shell_276_tilt.pdf} \\
\\
\\
{\color{red}Ra_f = 2700}\\
 % Ra_f = 2750 \\
\includegraphics*[width=90mm]{Figures/sph_shell_272_tilt.pdf} \\
\end{array}
\]
\end{center}
\caption{
Time evolution of dipole tilt angle for $Ra_f = 1500$ (upper panel) and {\color{red}$Ra_f = 2700$} 
% $Ra_f = 2750$
(lower panel).
}
\label{Fig:more_cases}
\end{figure}

{\color{red}
% Nishikawa and Kusano (2008) \cite{Nishikawa:2008} made investigations similar to those made in the present study.
% They investigated energy transfers by magnetic induction by equatorially symmetric and antisymmetric flow and magnetic diffusivity for equatorially symmetric and antisymmetric components of the magnetic field. 
Nishikawa and Kusano (2008) % \cite{Nishikawa:2008} 
investigated energy transfers to $\bvec{B}^s$ and 
 $\bvec{B}^a$ by $\bvec{u}^s$ and 
 $\bvec{u}^a$ and the magnetic diffusion in the induction equation.
}
They concluded that the magnetic induction by $\bvec{u}^a$ increases during polarity reversals. 
{\color{red}
% There are many differences between models by Nishikawa and Kusano (2008), such as compressibility of the fluid, magnetic boundary condition, dimensionless numbers, and the equation which is focusing. 
It should be noted that Nishikawa and Kusano (2008) adopted compressbility of core fluid and that the magnetic boundary condition and dimensionless numbers are different from those in the present dynamo model.
}
Especially, Nishikawa and Kusano (2008) adopted larger magnetic Prandtl numbers ($Pm = 10 \sim 15$). 
{\color{green}
The role of the Lorentz force increases with the magnetic Prandtl number $Pm$ in the dynamics of the fluid motion.
}
The difference of the magnetic Prandtl number can change the most important term for the polarity reversal process. 
% Lorentz force?

{\color{green}
In the present study, we mainly investigate the dynamics and energetics of the flow during polarity reversals. 
Now, we discuss the process of the magnetic field generation during polarity reversals. 
At the beginning of a reversal process, the amplitude of the dipole component decreases with decreasing the energy transfer from $\bvec{u}^s$ by the Lorentz force. 
When the kinetic energy for $\bvec{u}^a$ increases by the advection and buoyancy, the axial dipole component decreases and intense radial magnetic field is generated around the warm upward flow near the tangent cylinder in the either hemisphere. 
The upward flow also goes out of the tangent cylinder and reaches near the CMB in low latitude. 
At the end of the reversal, the warm upward flow comes out of the tangent cylinder. 
The flow can intensify the convection columns which generate the magnetic field. 
Consequently, the dipolar magnetic field with the opposite polarity increases in the outside of tangent cylinder with decreasing the equatorially antisymmetric flow and temperature. 

The result in the present study is consistent with a process of polarity reversal proposed by Wicht and Olson (2004), %\cite{Wicht:2004} 
although it indicates that the strong hot plume is generated in either one hemisphere during polarity reversals. Another difference from the results by Wicht and Olson (2004) is that the intense magnetic field generated inside the tangent cylinder changes its direction frequently during the polarity reversal.
}

% 以下の部分は上記とかぶっているように思います。 Yes, I rewrite them.
%Wicht and Olson (2004) showed a schematic diagram of the evolution for the zonal mean of temperature and current density during a polarity dipole reversal. 
%They pointed out that strong upwelling flow inside or around the tangent cylinder induces the reversal magnetic field and that the meridional circulation advects the reversed magnetic field. 
%The result in the present study is consistent with the process proposed by Wicht and Olson (2004), although it indicates that the strong hot plume is generated either one hemisphere during polarity reversals. 
%This hot plume is mainly generated inside the tangent cylinder, and strong zonal flow is induced along with the plume to satisfy thermal wind balance. 
%The hot plume and intense zonal flow sometimes go  out of the tangent cylinder. 
{\color{red}
% The intense zonal flow can generate intense zonal toroidal magnetic field only in one hemisphere, and the upwelling flow can generate the poloidal magnetic field with opposite polarity from the original dipolar field. 
The intense equatorially antisymmetric zonal flow can generate intense equatorially symmetric zonal toroidal magnetic field with satisfying the thermal wind balance, and then columnar convective flow can generate the poloidal magnetic field with opposite polarity from the original dipolar field.
}
% 下記をこの結果から言及できるでしょうか? Yes.
{\color{green}
% We expect that the poloidal magnetic field induced by conversing and upward flow motion along with the plumes from the bottom of the outer core. 
The poloidal magnetic field is likely to be induced by conversing and upward flow motion along with the plumes from the bottom of the outer core.
This polodal magnetic field is not axisymmetric when the plume goes to the outside of the tangent cylinder. 
Consequently, the generated field expands to global spherical shell (i.e.\ axisymmetric) to construct the reversed dipolar magnetic field.
We focus on the global energy transfer in the present study. 
Investigations of detailed reversal processes are required as a future study.
}

We conclude that increase of the equatorially antisymmetric flow with the axial symmetry is the primary cause of the polarity reversal in the present geodynamo model. 
However, it is still difficult to evaluate how much equatorially antisymmetric flow is required to initiate a reversal or an excursion quantitatively. 
We need more examples of polarity reversals to investigate statistically, and we also need more simulations with dipolar dominant field and polarity reversals for different parameters, especially lower Ekman and magnetic Prandtl numbers. 
To perform numerical simulations with a much smaller Ekman number, much higher spatial resolution or some sub-grid scale (SGS) model is required. 
Aubert (2019) pointed out that the hyperdiffusivity is feasible to model a turbulence process for the geodynamo modeling, because the role of turbulence is much smaller to the fluid dynamics in the core. 
However, the present study suggests that the inertia term can play an important role to drive the equatorially antisymmetric flow with axial symmetry during polarity reversals. 
We need further investigation to clarify which length scales of flow are important to generate the equatorially antisymmetric flow with axial symmetry and whether or not a more sophisticated model is required to represent the process to control this flow component.

% \begin{figure}[ht]
% \begin{center}
% \[
% \begin{array}{cc}
% Raf = 1500 & Raf = 2750 \\
% %\includegraphics*[width=60mm]{Figures/stable.pdf} &
% %\includegraphics*[width=60mm]{Figures/reversal.pdf} \\
% \includegraphics*[width=60mm]{Figures/sph_shell_276_tilt.pdf} &
% \includegraphics*[width=60mm]{Figures/sph_shell_272_tilt.pdf} \\
% \end{array}
% \]
% \end{center}
% \caption{
% Time evolution of dipole tilt angle in the cases with $Raf = 1500$ (left panel) and $Raf = 2750$ (right panel).
% }
% \label{Fig:more_cases}
% \end{figure}
%
