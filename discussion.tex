\section{Discussion}
\label{section:discussion}

%We restarted numerical simulations a few times from the identical snapshot using the same dimensionless numbers and the time step $\Delta t$.
{\color{teal}
We performed a few numerical simulations from the identical snapshot using the same dimensionless numbers and the time step, $\Delta t$.
}
However, we obtained results different from the original one; the solutions are not exactly identical
as found in the evolution of the dipole tilt angle after a few magnetic diffusion time (see Fig.~\ref{fig:dipole_tilt_retries}). 
Calypso chooses the fastest algorithms for the Legendre and Fourier transforms at the initialization process in numerical simulations. 
The different order of summation can change the last digit of data. 
As a result, numerical errors accumulate with the time integration. 
We found, however, that polarity reversals occur in the same way, and that the statistical results are similar to the original one.
{\color{blue} 
When we performed simulation under the same spherical harmonic transform algorithm and parallelization on the same computer system, we obtained identical solutions with a dipole reversal. In addition, the time average of the kinetic and magnetic energies in the all retried cases in the result section are up to 0.4 times of the standard deviation of the average kinetic and magnetic energies throughout the simulation. 
%In addition, we continues the simulation to more than 20 times of the magnetic diffusion times (see Fig.~\ref{fig:sph_shell_275_full}), and there is no significant difference in the occurrence of the dipole reversals and averaged kinetic and magnetic energies.
{\color{magenta}
%In addition, we continued the simulation to more than 20 time of the magnetic diffusion times from each retried case (see Fig.~\ref{fig:sph_shell_275_full}), and there is no significant difference in the occurrence of the dipole reversals and averaged kinetic and magnetic energies.
We further continued the numerical simulation up to more than 20 times of the magnetic diffusion time from each retried case (see Fig.~\ref{fig:sph_shell_275_full}).
We found that there is no significant difference in the occurrence of the dipole reversals and averaged kinetic and magnetic energies.
}
% Hence, we consider that the present results are substantial and feasible and that the reversal does not initiated by the numerical errors.
{\color{teal}
Hence, we consider that the present results are robust and that polarity reversals are not initiated by the numerical errors.
}
}

%The dimensionless numbers adopted in the present study are based on those used by Sreenivasan et al.\ (2014). % \cite{Sreenivasan:2014}. 
{\color{teal}
The dimensionless numbers adopted in the present study are based on those used by Sreenivasan et al.\ (2014) who obtained Earth-like dipole reversals with the Rayleigh number, $Ra_f = \alpha g_o \beta_o D^2 / \Omega \nu = 2700$.
}
{\color{red} We also performed dynamo simulations with two different flux Rayleigh numbers $Ra_{f} = 1500$ and 2700. The evolution of the dipole tilt at the outer boundary for these cases is shown in Fig.~\ref{Fig:more_cases}, and the $Rm$, $f_{dip}$, $\ell_{l}$, and $Ro_{\ell}$ in the stable and reversal periods are listed in Table \ref{table:average_dipolarity_2}.
}

\begin{table}[t]
\caption{Magnetic Reynolds number $Rm$, dipolarity $f_{\rm dip}$, 
{\color{blue} typical horizontal wave number for the convection $\ell_{l}$}
and local Rossby number $Ro_{\ell}$ in the spherical shell.}
\label{table:average_dipolarity_2}
\renewcommand{\arraystretch}{1.3} % 行間を1.5倍に
\begin{tabular}{cc|cccc}
$Ra_f$ & & $Rm$ & $f_{\rm dip}$ & 
  ${\ell}_{l}$ & $Ro_{\ell}$ \\ \hline
2700 & \mbox{Stable} & $426.4 \pm 168.7 $ &
                $0.109 \pm 0.054$ &
                $8.402 \pm 0.63$ & 
                $0.137$ \\
& \mbox{Reverse} & $368.2 \pm 144.0 $ &
                 $0.019 \pm 0.016$ &
                 $8.289 \pm 0.70$ & 
                 $0.143$\\ \hline
1500 & \mbox{Stable} & $304.7 \pm 21.4 $ &
                $0.195 \pm 0.040$ &
                $8.560 \pm 0.70$ & 
                $0.100$ \\
& \mbox{Reverse} &  \rule[0.5ex]{2em}{0.4pt} &
                  \rule[0.5ex]{2em}{0.4pt} &
                  \rule[0.5ex]{2em}{0.4pt} & 
                  \rule[0.5ex]{2em}{0.4pt} \\ \hline\end{tabular}
\end{table}
%
As shown in Fig.~\ref{Fig:more_cases}, however, we obtained a solution of multipolar dominated dynamo for $Ra_f = 2700$.
%{\color{teal}
%defined in (\ref{eq:dimensionless_numbers}).
%}
%, although Sreenivasan et al.\ (2014) obtained Earth-like dipole reversals with the same Rayleigh number (it is noted that Sreennivasan et al.\ (2014) defined the Rayleigh number as $Ra_f = \alpha g_o \beta_o D^2 / 2\Omega \nu$).
The difference of the behavior of reversals between the simulations by the present study and that by Sreenivasan et al. (2014) with the same $Ra_{f}$ is likely to originate from the different thermal boundary condition at the inner boundary. 
We fixed the heat flux at the inner boundary, while Sreenivasan et al.\ (2014) fixed the temperature at the inner boundary. 
%The uniformly fixed temperature boundary condition forces the temperature to be equatorially symmetric at the inner boundary, although the temperature can be equatorially antisymmetric away from the inner boundary.
{\color{teal}
The uniform temperature boundary condition at the ICB forces the temperature to be equatorially symmetric there, although the temperature can be equatorially antisymmetric away from the ICB.
}
Consequently, Sreenivasan et al.\ (2014) required a larger Rayleigh number to give rise to the equatorially antisymmetric temperature and flow in turn. 
However, no polarity reversal can be found in the cases of $Ra_f = 1500$ in the present study and $Ra_f = 1620$ in Sreenivasan et al.\ (2014) (See {\color{magenta}upper} panel of Fig.~\ref{Fig:more_cases}). 
%(See left panel of Fig.~\ref{Fig:more_cases}). 
The result suggests that the thermal boundary condition at the ICB does not have a significant effect on the lower bound of $Ra_f$ to give rise to the polarity reversal.

\begin{figure}[ht]
\begin{center}
\[
\begin{array}{c}
Ra_f = 1500  \\
\includegraphics*[width=90mm]{Figures/sph_shell_276_tilt.pdf} \\
\\
\\
{\color{red}Ra_f = 2700}\\
 % Ra_f = 2750 \\
\includegraphics*[width=90mm]{Figures/sph_shell_272_tilt.pdf} \\
\end{array}
\]
\end{center}
\caption{
Time evolution of dipole tilt angle for $Ra_f = 1500$ (upper panel) and {\color{red}$Ra_f = 2700$} 
% $Ra_f = 2750$
(lower panel).
}
\label{Fig:more_cases}
\end{figure}

Nishikawa and Kusano (2008) % \cite{Nishikawa:2008} 
investigated energy transfers to $\bvec{B}^s$ and $\bvec{B}^a$ by $\bvec{u}^s$ and $\bvec{u}^a$ and the magnetic diffusion in the induction equation.
{\color{teal}
They showed a process for the growth of the antisymmetric flow during polarity reversals, which differs from our model.
Specifically, during the stable phase, only the energy of the equatorially antisymmetric magnetic field is converted into the equatorially antisymmetric flow. However, during the reversal phase, the energy of the equatorially symmetric magnetic field is also converted into antisymmetric flow. Consequently, the energy conversion to the antisymmetric flow increases compared to the stable phase.
% For the stable phase, the energy of the antisymmetric magnetic field generated by the symmetric flow is converted to the antisymmetric flow, whereas flow energy is always transferred to the magnetic field regardless of the equatroial symmetry in our model.
% In contrast, for the reversal phase, the symmetric magnetic field is converted to antisymmetric flow because the symmetric magnetic field generation rate by the symmetric flow is higher than during the stable phase. Consequently, the energy conversion to antisymmetric flow becomes higher than during the stable phase, which modifies the equatorial symmetry of the flow.
}
% They concluded that the magnetic induction by $\bvec{u}^a$ increases during polarity reversals. 
It should be noted that Nishikawa and Kusano (2008) adopted compressbility of core fluid and that the magnetic boundary condition and dimensionless numbers are different from those in the present dynamo model.
Especially, Nishikawa and Kusano (2008) adopted larger magnetic Prandtl numbers ($Pm = 10 \sim 15$). 
The role of the Lorentz force increases with increase of the magnetic Prandtl number $Pm$ in the dynamics of the fluid motion.
The difference in the magnetic Prandtl number can change the most important term for the polarity reversal process. 

%The result in the present study is consistent with a process of polarity reversal proposed by Wicht and Olson (2004), %\cite{Wicht:2004} 
%although it indicates that the strong hot plume is generated in either one hemisphere during polarity reversals (See Figure 3(d). 
% Another difference from the results by Wicht and Olson (2004) is that the intense magnetic field generated inside the tangent cylinder changes its direction frequently during the polarity reversal.

%%  Need elabolation
{\color{teal}
The result in the present study is consistent with a process of polarity reversal proposed by Sarson and Jones (1999) and Wicht and Olson (2004).
However, a single strong hot plume is generated in one hemisphere during polarity reversals in our result, while multiple plumes are found in Wicht and Olson (2004).
{\color{red}
Additionally, our model shows that both the flow and magnetic field have finer structures and more complex time variations (see Animation 2).
}
% Another difference is that the magnetic field inside the tangent cylinder changes its direction several times during a polarity reversal in Wicht and Olson (2004).
On the other hand, flow fluctuations associated with buoyancy surges inside the tangent cylinder cause polarity reversals in the model of Sarson and Jones (1999), and the process is very similar to ours.
It should be pointed out, however, that Sarson and Jones (1999) employed the two-mode approximation (Jones et al., 1995) in geodynamo simulations.
This suggests that it would be difficult to investigate energy transfer in relation to polarity reversals even if there is a similarity to three-dimensional models.
}

% We conclude that increase of the equatorially antisymmetric flow with the axial symmetry is the primary cause of the polarity reversal in the present geodynamo model. 
% However, it is still difficult to evaluate how much equatorially antisymmetric flow is required to initiate a reversal or an excursion quantitatively. 

{\color{red}
Sprain et al. (2019) proposed Quality of Paleomagnetic Modeling criteria, $Q_{\rm PM}$ to asses semblance of the spatial and temporal variations of the magentic field between solutions of numerical dynamo simulations and the geomagnetic field using the five parameters (inclination anomaly, virtual geomagnetic pole dispersion at the equator, latitudinal variation in virtual geomagnetic pole dispersion, normalized width of virtual dipole moment distribution, and dipole field reversals) of the paleomagenetic records. Sprain et al. (2019) set five criteria from 1 to 5 for $Q_{\rm PM}$ and investigated 46 dynamo models, but the maximum $Q_{\rm PM}$ was 3. Meduri et al. (2021) also investigated the reproducivity of the paleomagnetic data in the thermally and compositionally driven dynamo simulations with Ekman number smaller than that by Sprain et al. (2019) using the $Q_{\rm PM}$ index. Meduri et al. (2021) obtained more Earth-like behavior ($Q_{\rm PM}$ = 4) in the chemically driven dynamo cases with lower Ekman number.
}

{\color{red}
%There is no case with the same dimensionless number in these studies, but the magetic Raynolds number in the present study $Rm = 364$ is very close to the Run C in Meduri et al. (2021), which has $Rm = 376$ and dipole reversal. 
{\color{teal}
The same set of dimensionless numbers adopted as in the present study was not used in Sprain et al. (2019) and Meduri et al. (2021).
However, the magnetic Reynolds number, $Rm = 364$, obtained in the present study is very close to $Rm = 376$ obtained in the Run C of Meduri et al. (2021) in which dipole reversals occurred.
}
We expect that our parameter regime is between Run C in Meduri et al. (2021) and Case 4 in Sprain et al. (2019). 
%The comparison with the geomagnetic field variation like $Q_{\rm PM}$ will be the further investigation of the present results.
}
{\color{teal}
A comparison of our results to the spatial and temporal variations of the geomagnetic field in terms of $Q_{\rm PM}$, for example, will be necessary in the future.}

%{\color{teal}}
%The present simulation is still far from the actual parameters for the Earth's outer core even considering the turbulent diffusivity. 
{\color{blue}
The parameters adopted in the present simulation are still far from the actual ones for the Earth's outer core even considering the turbulent diffusivity. 
In addition, the amplitude of the inertia in the outer core is expected to be $10^{-5}$ times smaller than that of the Coriolis force. 
The forces contributing to the energy transfer correspond to the second order of force balance which is the rest of the Magnetic-Alchimedic-Coriolis (MAC) balance. 
Furthermore, we investigated the perturbation from the average value of energy fluxes. 
%We expect that the inertia can have a large role in the perturbation of the second order force balance, but we need more investigation the order of the second order force balance in the Earth's core.
{\color{teal}
We consider that the inertia can have a large role in the perturbation of the second order force balance, which must be investigated much more.
}
}

%For these further investigations, we also need more examples of polarity reversals to investigate statistically, and we also need more simulations with dipolar dominant field and polarity reversals for different parameters, especially lower Ekman and magnetic Prandtl numbers. 
{\color{teal}
For further statistical investigations of polarity reversals, we need more examples of numerical dynamos with dipolar dominant field as well as polarity reversals for different parameters, especially lower Ekman and magnetic Prandtl numbers.
}
To perform numerical simulations with a much smaller Ekman number, much higher spatial resolution or some sub-grid scale (SGS) model is required. 
Aubert (2019) pointed out that the hyperdiffusivity is feasible to model a turbulence process for the geodynamo modeling, %because the role of turbulence is much smaller to the fluid dynamics in the core.
{\color{teal}
because turbulence plays an unimportant role in the core dynamics.
}
However, the present study suggests that the inertia term can play an important role to drive the equatorially antisymmetric flow with axial symmetry during polarity reversals.
%
%We need further investigation to clarify which length scales of flow are important to generate the equatorially antisymmetric flow with axial symmetry and whether or not a more sophisticated model is required to represent the process to control this flow component.

{\color{teal}
We conclude that increase of the equatorially antisymmetric flow with the axial symmetry is the primary cause of the polarity reversal in the present geodynamo model. 
However, it is still difficult to evaluate how much equatorially antisymmetric flow is required to initiate a reversal or an excursion quantitatively.
We need further investigation to clarify which length scales of flow are important to generate the equatorially antisymmetric flow with axial symmetry and whether or not a more sophisticated model is required to represent the process to control this flow component.
}

% \begin{figure}[ht]
% \begin{center}
% \[
% \begin{array}{cc}
% Raf = 1500 & Raf = 2750 \\
% %\includegraphics*[width=60mm]{Figures/stable.pdf} &
% %\includegraphics*[width=60mm]{Figures/reversal.pdf} \\
% \includegraphics*[width=60mm]{Figures/sph_shell_276_tilt.pdf} &
% \includegraphics*[width=60mm]{Figures/sph_shell_272_tilt.pdf} \\
% \end{array}
% \]
% \end{center}
% \caption{
% Time evolution of dipole tilt angle in the cases with $Raf = 1500$ (left panel) and $Raf = 2750$ (right panel).
% }
% \label{Fig:more_cases}
% \end{figure}
%
