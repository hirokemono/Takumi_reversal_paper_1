\section{Discussion}
\label{section:discussion}

We restarted numerical simulations a few times from the identical snapshot using the same dimensionless numbers and the time step $\Delta t$.
However, we obtained results different from the original one; the solutions are not exactly identical
as found in the evolution of the dipole tilt angle after a few magnetic diffusion time (see Fig.~\ref{fig:dipole_tilt_retries}). 
Calypso chooses the fastest algorithms for the Legendre and Fourier transforms at the initialization process in numerical simulations. 
The different order of summation can change the last digit of data. 
As a result, numerical errors accumulate with the time integration. 
{\color{red}
% However, the time average the results are almost identical, and the dipole reversal happens in the both original and retried cases.
We found, however, that polarity reversals occur in the same way, and that the statistical results are similar to the original one.
% We consider that the present results have a limitation of a numerical modeling, but that these results are still feasible.
Hence, we consider that the present results are substantial and feasible.
}

The dimensionless numbers adopted in the present study are based on those used by Sreenivasan et al.\ (2014) \cite{Sreenivasan:2014}. 
As shown in Fig.~\ref{Fig:more_cases}, however, we obtained a non-dipolar magnetic field solution for $Ra_f = 2700$, although Sreenivasan et al.\ (2014) \cite{Sreenivasan:2014} obtained Earth-like dipole reversals with the same Rayleigh number (it is noted that Sreennivasan et al.\ (2014) defined the Rayleigh number as $Ra_f = \alpha g_o \beta_o D^2 / 2\Omega \nu$).
The difference is likely to originate from the different thermal boundary condition at the inner boundary. 
We fixed the heat flux at the inner boundary, while Sreenivasan et al.\ (2014) fixed the temperature at the inner boundary. 
{\color{red}
The uniformly fixed temperature boundary condition forces the temperature to be equatorially symmetric at the inner boundary, although the temperature can be equatorially antisymmetric away from the inner boundary.
}
Consequently, Sreenivasan et al.\ (2014) \cite{Sreenivasan:2014} required a larger Rayleigh number to give rise to the equatorially antisymmetric temperature and flow in turn. 
% 以下,よくわかりませんでした...
{\color{green}
However, we expect that the effects of the homogeneous thermal boundary condition is not significant because no reversal is obtained for $Ra_f = 1500$ in the present study and for $Ra_f = 1620$ in Sreenivasan et al.\ (2014) (See left panel of Fig.~\ref{Fig:more_cases}).
}

\begin{figure}[ht]
\begin{center}
\[
\begin{array}{cc}
Ra_f = 1500 & Ra_f = 2750 \\
%\includegraphics*[width=60mm]{Figures/stable.pdf} &
%\includegraphics*[width=60mm]{Figures/reversal.pdf} \\
\includegraphics*[width=60mm]{Figures/sph_shell_276_tilt.pdf} &
\includegraphics*[width=60mm]{Figures/sph_shell_272_tilt.pdf} \\
\end{array}
\]
\end{center}
\caption{
Time evolution of dipole tilt angle for $Ra_f = 1500$ (left panel) and $Ra_f = 2750$ (right panel).
}
\label{Fig:more_cases}
\end{figure}

Nishikawa and Kusano (2008) \cite{Nishikawa:2008} made investigations similar to those made in the present study.
They investigated energy transfers by magnetic induction by equatorially symmetric and antisymmetric flow and magnetic diffusivity for equatorially symmetric and antisymmetric components of the magnetic field. 
They concluded that the magnetic induction by equatorially antisymmetric flow increases during the reversals. 
There are many differences between models by Nishikawa and Kusano (2008), such as compressibility of the fluid, magnetic boundary condition, dimensionless numbers, and the equation which is focusing. Especially, Nishikawa and Kusano (2008) adopted a larger magnetic Prandtl number. 
The difference of the magnetic Prandtl number can change the most important term for the reversal processes.

Now, we will discuss the processes of the magnetic field generation during the reversals. 
Wicht and Olson (2004) \cite{Wicht:2004} showed a schematic diagram of the evolution of the zonal mean of temperature and current density during the dipole reversal. 
They suggested that strong upwelling flow inside or around the tangent cylinder induces the reversal magnetic field and that the meridional circulation adevects the reversed magnetic field. 
The results in the present study consists with the Wicht and Olson's proposed model, but the present results suggested that the strong hot plume is generated either one hemisphere during reversals. 
This hot plume is mainly generated inside the tangent cylinder, and strong zonal flow is generated along with the plume to satisfy thermal wind balance. 
This hot plume and intense zonal flow sometimes go to the outside of the tangent cylinder. 
This intense zonal flow can generate intense toroidal magnetic field only one hemisphere, and the upwelling flow can generate the poloidal magnetic field with opposite direction from the original dipolar field. 
We expect that the poloidal field induced by the radial motion along with the plume is not axisymmetric when the plume goes to the outside of the tangent cylinder. 
However, the generated field will expands to globally. 
We focus on the global energy transfer in the present study. 
Investigations of detailed reversal processes are required as the future studies.

We conclude that the increasing the equatorially antisymmetric and axissymetric flow is the primary cause of the dipolar reversal in the present geodynamo model. 
However, it is still not difficult to evaluate how much equatorially antisymmetric component is required to initiate reversal or excursion quantitatively. 
We need more reversal to investigate statistically, and we also need more simulations with dipolar dominant field and its reversals in different parameters, especially lower Ekman and magnetic Prandtl numbers. 
To perform numerical simulations with a much smaller Ekman number, much higher spatial resolution or some sub-grid scale (SGS) model is required. 
Aubert {\color{red}(2019)} suggested that the hyperdiffusivity is feasible to model a turbulence process for the geodynamo modeling, because the role of turbulence is much smaller to the dynamics of the fluid in the core. 
However, the present study suggests that the inertia term can play an important role to drive equatorially antisymmetric and axisymmetric flow during the dipole reversal. 
We need more investigation to clarify which length scales of flow are important to generate the equatorially antisymmetric and axisymmetric flow energy and if more sophisticated model is required to represent the processes to control this flow component.

% \begin{figure}[ht]
% \begin{center}
% \[
% \begin{array}{cc}
% Raf = 1500 & Raf = 2750 \\
% %\includegraphics*[width=60mm]{Figures/stable.pdf} &
% %\includegraphics*[width=60mm]{Figures/reversal.pdf} \\
% \includegraphics*[width=60mm]{Figures/sph_shell_276_tilt.pdf} &
% \includegraphics*[width=60mm]{Figures/sph_shell_272_tilt.pdf} \\
% \end{array}
% \]
% \end{center}
% \caption{
% Time evolution of dipole tilt angle in the cases with $Raf = 1500$ (left panel) and $Raf = 2750$ (right panel).
% }
% \label{Fig:more_cases}
% \end{figure}
%
