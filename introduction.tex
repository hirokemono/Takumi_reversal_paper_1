\newpage
\section{Introduction}
\label{section:introduction}
The geomagnetic field has been sustained over 3.5 billion years, and the direction of the current geomagnetic dipole component is nearly aligned with the Earth's rotation axis. 
The past amplitude and direction of the geomagnetic field can be estimated by the paleomagnetic observations using igneous or sedimentary rocks. 
The paleomagnetic observations and observation of the magnetic anomalies reveals direction of the geomagnetic field has been reversed frequently in the geological time scales. 
These results have strongly supported that the geomagnetic field is sustained by the flow motion of the liquid iron alloy in the Earth's outer core, so called the geodynamo process. 

% To understand the geodynamo processes, 
Numerical simulations {\color{red} of geodynamo} have 
% had large 
{\color{red} been playing an important} role to understand the geodynamo processes and flow dynamics of the Earth's outer core.
After Glatzmaier and Roberts (1995) \cite{GR:1995} and Kageyama et al.\ (1995) \cite{Kageyama:1995}, many magnetohydrodynamics (MHD) simulations have been successfully performed to represent the characteristics of the geomagnetic field. 
The reversal of the axial dipole component has been also represented in the geodynamo simulations (e.g. Glatzmaier and Roberts, 1995 \cite{GR:1995}, Sarson and Jones, 1999\cite{Sarson:1999}; Olson et al., 2011; Sreenivasan et al., 2014).

The convection in the outer core is expected to be dominated by the geostrophic balance, which balances the Coriolis force and pressure gradient. 
Busse (1970) \cite{Busse:1970} suggested that the convection in the rotating spherical shell mainly occurs outside of the tangent cylinder, which is a imaginary cylinder with the radius of the inter core, and that the convection is characterized by multiple convection columns along with the rotation axis. 
These convection is symmetric with respect to the equator, and the Lorentz force by is also symmetric with respect to the equator if the magnetic field only has the anti-symmetric with respect to the equator such as the axial dipole field.
However, it is suggested that the equatorial symmetry of the convection in the outer core is broken and equatorialy symmetric component of the magnetic field increases during the dipole reversals. Based on the paleomagnetic field observations for the last 150 million years, the amplitude ratio of equatorially ant-symmetric component of the non-axial dipole component to the total non-axial dipole component of the geomagnetic field has anti-corrilation with the reversal occerency (Coe and Glatzmaier, 2006 \cite{Coe:2006}). In nuerical simulaions, Glatzmaier et al.\ (1999) \cite{Glatzmaier:1999} shows that the equatorially anti-symmetric components are dominant when the axial dipole field is stably generated in their geodynamo simulations. For flow motion, Wicht and Olson (2004) concluded that the reversed axisymmetric toroidal electric current density ({\it i.e.} axisymmetric poloidal magnetic field) is generated near the outer boundary and tangent cylinder, which is an imaginary cylinder adjoint with the inner core boundary, by the plume with upwelling flow and that equaitorially anti-symetricc meridional circulation advects the reversed zonal current density to the whole outer core during the reversals. Li et al.\ (2009) \cite{Li:2002} also represented the equaitorial symmetry of the both meridional circulation and zonal flow is collapsed during the dipole reversal in their numerical dynamo model.

The control factors for the dipole reversal have also been % studied by 
{\color{red} examined in} previous studies. 
Glatzmaier et al.\ (1999) \cite{Glatzmaier:1999} performed thermally driven dynamo simulations with changing heat flux patterns at the outer boundary of the spherical shell and found that more reversals occurs in the case with smaller heat flux at high latitude ({\it i.e.} the outer core boundary). 
Sreenivasan et al.\ (2014) \cite{Sreenivasan:2014} performed dynamo simulations with changing Rayleigh number and showed that occurrence of the dipole reversal increases with the Rayleigh number. 
Olson and Christensen (2006) \cite{Olson:2006} shows that the generated magnetic fields transit from dipole dominant field to multipolar magnetic field with increasing the Rayleigh number and Earth-like dipole reversal is represented the cases with the Rayleigh number which is transferring from the dipolar to multipolar regime. 
In addition, they also suggested that inertia can have a large role to drive the dipole reversals by scaling between dipole field strength and local Rossby number. 
Nishikawa and Kusano (2008) \cite{Nishikawa:2008} focused the energy transfer by the magnetic induction equation with splitting the equatorially symmetric and antisymmetric components. 
Nishikawa and Kusano (2008) suggests that the direction of the energy transfer changes between stable dipole period and during reversals. 
The energy of the symmetric components of the magnetic energy transfers to the anti-symmetric components of the kinetic energy during the reversals, while the energy transfers from anti-symmetric kinetic energy to the symmetric components of the magnetic energy in the stable dipole period.

In the present study, we focus on how the energies of equatorially symmetric and anti-symmetric component of the flow are transferred by the buoyancy, advection, and Lorentz force. 
We perform MHD dynamo simulations in a rotating spherical shell modeled on the Earth's outer core and evaluate the work of the buoyancy, inertia term, and Lorentz force for the equatorially symmetric and anti-symmetric components averaged over the spherical shell. 
We obtained 10 reversals and investigate how energy fluxes changes around the dipole reversals. We will explain the models of the present dynamo simulation and describe the energy equations for the equatorially symmetric and antisymmetric componennts of the kinetic energy. 
In Section 3, we will show the results of the simulation and data analysis of the work of forces. In section 4, we will discuss the comparison with the results of previous studies and magnetic field generation processes during the dipole reversal in the present simulation. 
Finally, conclusions will be described in Section 5.
