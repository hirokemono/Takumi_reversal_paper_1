\section{Introduction}
The geomagnetic field has been sustained over 3.5 billion years, and the direction of the current geomagnetic dipole component is nearly alligned with the Earth's rotation axis. The past amplitude and direction of the geomagnetic field can be estimated by the paleomagnetic observations using igneous or sedimentary rocks. The paleomagnetic observations and observation of the magnetic anomalies reveals direction of the geomagnetic field has been reversed frequently in the geological time scales. These results have strongly supported that the geomagnetic field is sustained by the flow motion of the liquid iron alloy in the Earth's outer core, so called geodynamo processes. 

To understand the geodynamo processes, numerical simulations have had large roles to understand the geodynamo processes and flow dynamics of the EArth's outer core. After Glatzmaier and Roberts (1995) \cite{Glatz:95} and Kageyama {\it et al.} (1995) \cite{Kageyama:95}, a number of magnetohydrodynamics (MHD) simulations have been successfully performed to represent the charactersitics of the geomagnetic field. The reversal of the axial dipole component has been also represented in the geodynamo simulations ({\it e.g.} Glatzmaier and Roberts (1995) \cite{Glatz:95}, Sarson and Jones,1999\cite{Jones:99}; Olson {it et al.}, 2011; and Sreenivasan {\it et al.}, 2014).

The convection of the outer core is expected to be dominated by the geostrophic balance,  which balances the Coriolis force and pressure gradient. Busse (1970) \cite{Busse:70} suggested that the convection in the rotating spherical shell mainly occurs outside of the tangent cylinder, which is a imaginally cylinder with the radius of the inter core, and that the convection is characterized by multiple convection columns along with the rotation axis. These convection is symmetric with respect to the equator. The Lorentz force is also have a large role in the dynamics of the outer core. The Lorentz force is also symmetric with respect to the equator if the magnetic field only has the anti-symmetric with respect to the equator such as the axial dipole field. 