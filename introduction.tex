\newpage
\section{Introduction}
\label{section:introduction}
The Earth has an intrinsic magnetic field which is dominated by a dipole component roughly aligned with the Earth's rotation axis. 
The amplitude and direction of the past geomagnetic field can be estimated by paleomagnetic observations using igneous or sedimentary rocks. Recent paleomagnetic observations have revealed that the geomagnetic field has been sustained for 4.2 billion years (Tarduno et al., 2020). % \cite{Tarduno:2020}).
In addition, the paleomagnetic observations and observations of the magnetic anomalies around the oceanic ridges have revealed that the direction of the geomagnetic field has frequently reversed in the geological time scale (e.g.\ Cande and Kent, 1995). %\cite{Cande:1995}, for example). 
These results strongly support that the geomagnetic field is maintained by a flow motion of the liquid iron alloy in the Earth's outer core, so called the geodynamo process.

% Numerical simulations of geodynamo have been playing an important role to understand the geodynamo processes and flow dynamics of the Earth's outer core.
{\color{red}
Numerical simulations of geodynamo have been playing an important role to understand the dynamo processes and magnetohydrodynamics (MHD) in the Earth's outer core.
}
After Glatzmaier and Roberts (1995) %\cite{GR:1995} 
and Kageyama et al.\ (1995), %\cite{Kageyama:1995}, 
many MHD dynamo simulations have been successfully performed to represent the characteristics of the geomagnetic field. 
The polarity reversal of the axial dipole component has also been represented in the geodynamo simulations (e.g.\ Glatzmaier and Roberts, 1995; 
% \cite{GR:1995}; 
Sarson and Jones, 1999;
% \cite{Sarson:1999}; 
Takahashi et al., 2007; 
% \cite{TMH:2007}; 
Olson et al., 2011 
% \cite{Olson:2011}; 
Sreenivasan et al., 2014). % \cite{Sreenivasan:2014}).

The convection in the outer core is likely to be dominated by the geostrophic balance, in which the Coriolis force is balanced by the pressure gradient.
Busse (1970) 
% \cite{Busse:1970} 
suggested that the convection in a rotating spherical shell occurs mainly outside the tangent cylinder, which is an imaginary cylinder with the radius of the inner core, and that the convection is characterized by multiple convective columns along with the rotation axis. 
The columnar helical flow in the anti-cyclonic convection columns generates the dipolar magnetic field by twisting magnetic field lines in the anti-cyclonic columns, and the zonal magnetic field line is extended with the cyclonic convection columns (Kageyama and Sato, 2017). % \cite{Kageyama:97c}). 
% These characteristics of the convection in the rotating spherical shell is symmetric with respect to the equatorial plane, and the Lorentz force is also symmetric with respect to the equatorial plane if the magnetic field only has the antisymmetric with respect to the equatorial plane such as the axial dipole field.
{\color{red}
These convective motions in a rotating spherical shell are characterized by symmetry with respect to the equatorial plane.
The Lorentz force is also nearly symmetric with respect to the equatorial plane if the magnetic field is dominantly antisymmetric with respect to the equatorial plane such as the axial dipole field.
}
However, it is suggested that the equatorial symmetry of the convection in the outer core is broken and equatorially symmetric components of the magnetic field increases during the polarity reversals both from the paleomagnetic observations and numerical simulations. % REFERENCE are in the following two sentences.
Based on the paleomagnetic observations for the last 150 million years, the amplitude ratio of equatorially antisymmetric component excluding axial dipole component to the total off-axis dipole component of the geomagnetic field is inversely correlated with the reversal occurrence (Coe and Glatzmaier, 2006). % \cite{Coe:2006}). 
Glatzmaier et al.\ (1999) % \cite{Glatzmaier:1999} 
showed that the equatorially antisymmetric components of the magnetic field are dominant when the axial dipole field is stably generated in their geodynamo simulations. 
The Earth-like magnetic dipole reversal is characterized by the stable dipole dominant magnetic field and spontaneous rapid reversal of the dipole component. 
Regarding flow motion, several numerical dynamo models with Earth-like polarity reversal represented a breakdown of the equatorial symmetry of the both meridional circulation and zonal flow  during the dipole reversal (Li et al., 2002; % \cite{Li:2002}; 
Wicht and Olson, 2004). % \cite{Wicht:2004}).
Wicht and Olson (2004) concluded that the reversed axisymmetric toroidal electric current (i.e.\ axisymmetric poloidal magnetic field) is generated near the outer boundary and tangent cylinder, which is an imaginary cylinder adjoint with the inner core boundary, by the plume with upwelling flow, and that equatorially antisymmetric meridional circulation advects the reversed zonal current to the whole outer core during the reversals.

The control factors for the polarity reversal have also been examined so far.
Glatzmaier et al.\ (1999) % \cite{Glatzmaier:1999} 
performed thermally driven dynamo simulations with changing heat flux patterns at the outer boundary of the spherical shell and found that more reversals occurs in the case of smaller heat flux at high latitude. %({\it i.e.} the outer core boundary).
{\color{red} % 合っている? OKです
% Parameter regimes of the geodynamo simulations with Earth-like dipole reversal can generally be found between the parameter regimes to sustain stable intense dipolar field without reversal and that to generate weak and periodically variable dipole field with small scale magnetic field (Christensen and Aubert, 2006 \cite{aubert:2006}; Driscoll and Olson, 2009 \cite{driscoll:2009}).
Parameters used in the geodynamo simulations with Earth-like polarity reversal can generally be found between the dynamo regime to sustain stable intense dipolar field without reversal and that to generate weak and periodically variable dipole field with small scale magnetic field (Christensen and Aubert, 2006; 
%\cite{aubert:2006}; 
Driscoll and Olson, 2009). % \cite{driscoll:2009}).
}
Sreenivasan et al.\ (2014) % \cite{Sreenivasan:2014} 
performed dynamo simulations with various Rayleigh numbers and showed that occurrence of the polarity reversal increases with increase of the Rayleigh number. 
Olson and Christensen (2006) % \cite{Olson:2006} 
showed that the generated magnetic fields change from dipole dominant field to multipolar magnetic field with increase of the Rayleigh number and Earth-like polarity reversal is represented in the cases with the Rayleigh number which is transferring from the dipolar to multipolar regime. 
In addition, they also pointed out that inertia can have a large role to give rise to polarity reversals by scaling between dipole field strength and local Rossby number.
{\color{green}
Nakagawa and Davies (2022) 
% \cite{Nakagawa:2022} 
also performed dynamo simulations with reversing dipole components and concluded that the role of the inertia is not negligible, even if QG-MAC (Quasi-Geostrophic with Magnetic, Archimedean and Coriolis) dynamics balance is dominant during the polarity reversal.
}
Nishikawa and Kusano (2008) 
% \cite{Nishikawa:2008} 
focused on the energy transfer in the magnetic induction equation separated into the equatorially symmetric and antisymmetric components. 
Nishikawa and Kusano (2008) mentioned that the direction of the energy transfer changes between stable dipole and reversals periods. 
The energy of the equatorially symmetric components of the magnetic energy transfers to the equatorially antisymmetric components of the kinetic energy during the reversals, while the energy transfers from equatorially antisymmetric kinetic energy to the equatorially symmetric components of the magnetic energy in the stable dipole period.

In the present study, we focus on how the energies of equatorially symmetric and antisymmetric flows are transferred by the buoyancy, advection, and Lorentz force. 
We perform MHD dynamo simulations in a rotating spherical shell modeled on the Earth's outer core and evaluate the work of the buoyancy, inertia term, and Lorentz force for the equatorially symmetric and antisymmetric components averaged over the spherical shell. 
% We obtained 12 reversals and investigate how energy fluxes changes around the dipole reversals. 
In Section \ref{section:method}, we will explain the models of the present dynamo simulation and describe the energy equations for the equatorially symmetric and antisymmetric flows. 
In Section \ref{section:results}, we will show the results of the simulation and analyze data in terms of the work by forces. 
In Section \ref{section:discussion}, we will discuss the comparison with the results of previous studies and magnetic field generation processes during the polarity reversal in the present simulation. 
Finally, conclusions will be described in Section \ref{section:conclusions}.
